\documentclass[12pt]{article}

\usepackage{setspace}

\usepackage{xcolor, amsmath, amsfonts, amssymb, graphicx, color, fancyhdr, lipsum, scalerel, stackengine, mathrsfs, tikz-cd, mdframed, enumitem}
\usepackage[amsthm]{ntheorem}
\usepackage[mathscr]{euscript}
%set margins
\usepackage[
top    = 1.0in,
bottom = 1.0in,
left   = 1.0in,
right  = 1.0in]{geometry}

%header stuff
\setlength{\headsep}{24pt}  % space between header and text
\pagestyle{fancy}     % set pagestyle for document
\lhead{ Problems from our reading course } % put text in header (left side)
\rhead{Nico Courts} % put text in header (right side)
\cfoot{\it p. \thepage}
\setlength{\headheight}{15pt}
\allowdisplaybreaks

%Set of Integers
\newcommand*{\Z}{
\mathbb{Z}
}
%Set of Natural Numbers
\newcommand*{\N}{
\mathbb{N}
}
%Set of Real Numbers
\newcommand*{\R}{
\mathbb{R}
}
%Set of Complex Numbers
\newcommand*{\C}{
\mathbb{C}
}
%Field
\newcommand*{\F}{
\mathbb{F}
}
%Rationals
\newcommand*{\Q}{
\mathbb{Q}
}

%Section break
\newcommand*{\brk}{
\rule{2in}{.1pt}
}

\DeclareMathOperator{\Aut}{Aut}

%raise that Chi!
\DeclareRobustCommand{\Chi}{{\mathpalette\irchi\relax}}
\newcommand{\irchi}[2]{\raisebox{\depth}{$#1\chi$}} 

%Image
\DeclareMathOperator{\im}{Im}

%Hom
\DeclareMathOperator{\Hom}{Hom}

%Ext
\DeclareMathOperator{\Ext}{Ext}

%Coker
\DeclareMathOperator{\coker}{coker}

%characteristic
\DeclareMathOperator{\ch}{char}

%restriction
\DeclareMathOperator{\Res}{Res_H}

%socle
%restriction
\DeclareMathOperator{\Soc}{Soc}

%induction
\DeclareMathOperator{\Ind}{Ind_H^G}

%fix tilde
\let\tilde\relax
\newcommand*{\tilde}[1]{\widetilde{#1}}

\newtheorem{lem}{Lemma}[subsection]
\newtheorem{thm}[lem]{Theorem}
\newtheorem{pro}[lem]{Problem}

\setenumerate[0]{label=(\alph*)}

%New environments for problem solving
\newenvironment{prob}{\par\smallskip
	\noindent\begin{mdframed}\small \begin{pro}}{\end{pro}\end{mdframed}\medskip}
\newenvironment{sol}{\noindent \textbf{Solution.} \,}{\\\hspace*{\fill}$\square$\medskip}

%\onehalfspacing
\begin{document}
%make the title page
\title{ Notes and Problems from My Research\vspace{-1ex}}
\author{Nico Courts}
\date{}
\maketitle

%begin problems

\section{Autumn 2018}
\subsection{Problems}
\begin{prob}
	Assume that $k$ is a field and let $K=k(t)$ (notice $K$ is a transcendental extension). Prove that $\Hom_k(K,k)\not\cong K$.
\end{prob}

\begin{sol}
	This is basically just a cardinality argument. I don't think it's particularly worth doing at this juncture.
\end{sol}

\begin{prob}
	Let $G$ be a finite group scheme (actually we need only assume that $G$ is a Frobenius algebra so that a module
	is injective if and only if it is projective). Prove that unless $M$ is projective, its projective dimension is 
	infinite. Conclude that $H^n(G,M)=0$ for $n>N$ implies that $M$ is projective.
\end{prob}
\begin{sol}
	Assume $M$ itself is not projective so that its minimal projective resolution is nontrivial and furthermore
	that it is finite. That is, let $P_i$ be projective modules such that
	\[0\to P_n\xrightarrow{f_n} P_{n-1}\to\cdots \xrightarrow{f_{1}}P_0\xrightarrow{f_0} M\to 0\]
	is a minimal length projective resolution of $M$ (notice here that $n\ge 1$). 
	
	Next consider the short exact sequence
	\[0\to P_n\xrightarrow{f_n}P_{n-1}\to \coker f_n\to 0\]
	since $P_n$ is projective (and thus injective!) this sequence splits and therefore $P_{n-1}\cong P_n\oplus\coker f_n.$
	But then consider the sequence
	\[0\to P_n\xrightarrow{g} P_{n-2}\to\cdots\xrightarrow{f_0}M\to 0\]
	where above we are using $P_{n-1}\supseteq P_n\cong f_n(P_n)$  and that
	$g=f_{n-1}|_{f_n(P_n)}$. This map is injective since $\ker f_{n-1}=\coker f_n$, which is disjoint
	from $f_n(P_n)\cong P_n$. Exactness everywhere else is evident since the maps are not effectively changed.

	But then the existence of this sequence contradicts the minimality of the original sequence, so
	no finite sequence can exist.
\end{sol}

\begin{prob}
	Establish the five-term exact sequence for spectral sequences.
\end{prob}
\begin{sol}
	I plan to return to this problem in the future. I have other priorities at the moment,
	but I will eventually return to cohomology and spectral sequences and this will be a good
	exercise at that point.
\end{sol}

\section{Winter 2019}
\subsection{Preparation: Waterhouse and G\"ortz \& Weddhorn}
Problems I worked from Waterhouse can be found in the appropriate file. 

\end{document}
