\documentclass[12pt]{article}

\usepackage{setspace}

\usepackage{amsmath, amsfonts, amssymb, graphicx, color, fancyhdr, lipsum, scalerel, stackengine, mathrsfs, tikz-cd, mdframed, enumitem, framed, adjustbox, bm, upgreek, x	color}
\usepackage[framed,thmmarks]{ntheorem}
\usepackage[mathscr]{euscript}



\setlength\fboxsep{15pt}


/home/nico/latex-includes/mydefs.tex

%Shade definitions
\theoremindent0cm
\theoremheaderfont{\normalfont\bfseries} 
\def\theoremframecommand{\colorbox[rgb]{.85,1,.85}}
\newshadedtheorem{defn}[thm]{Definition}


%Replacement for the old geometry package
\usepackage{fullpage}

%Put page breaks before each part
\let\oldpart\part%
\renewcommand{\part}{\clearpage\oldpart}%


\newcommand*{\Coalgk}{\mathbf{Coalg}_k}
\newcommand*{\Coalg}{\mathbf{Coalg}}
\newcommand*{\g}{\mathfrak{g}}

%Center each figure by default
\makeatletter
\g@addto@macro\@floatboxreset{\centering}
\makeatother

%header stuff
\setlength{\headsep}{24pt}  % space between header and text
\pagestyle{fancy}     % set pagestyle for document
\lhead{Notes on Hopf Algebras} % put text in header (left side)
\rhead{Nico Courts} % put text in header (right side)
\cfoot{\itshape p. \thepage}
\setlength{\headheight}{15pt}
\allowdisplaybreaks

%Set of Integers
\newcommand*{\Z}{
\mathbb{Z}
}
%Set of Natural Numbers
\newcommand*{\N}{
\mathbb{N}
}
%Set of Real Numbers
\newcommand*{\R}{
\mathbb{R}
}
%Set of Complex Numbers
\newcommand*{\C}{
\mathbb{C}
}
%Rationals
\newcommand*{\Q}{
\mathbb{Q}
}



%Image
\DeclareMathOperator{\im}{Im}


%fix tilde
\let\tilde\relax
\newcommand*{\tilde}[1]{\widetilde{#1}}

% Enumerate will automatically use letters (e.g. part a,b,c,...)
\setenumerate[0]{label=(\alph*)}

\begin{document}
%make the title page
\title{Hopf Algebras\vspace{-1ex}}
\author{A course by: Prof. James Zhang\\
Notes by: Nico Courts}
\date{Spring 2019}
\maketitle

\renewcommand{\abstractname}{Introduction}
\begin{abstract}
	These are the notes I took in class during the Winter 2019 topics course
	\textit{Math 582H - Hopf Algebras} at University of Washington, Seattle. 
	
	The course description follows:

	\brk

	This course is an introductory course on homological algebra. We will be following the 
	book \textit{An Introduction to Homological Algebra} by Charles Weibel. We will be
	covering the following topics:

\begin{itemize}
	\item Chain complexes, homotopies, homology and long exact sequence in homology
	\item Resolutions, derived functors, Ext and Tor. Koszul complexes
	\item Group (co)homology
	\item Triangulated and derived categories
	\item Spectral sequences or open topic depending on the class interests
\end{itemize}
\end{abstract}

\section{April 1, 2019}
If you don't know what a symmetric tensor category is, today is going to be a three 
star day. Max is 5.

\subsection{Overview}
We are shooting to understand two conjectures:

\begin{conj}[Etingof-Ostrik `04]
	If $A$ is a finite dimentional Hopf algebra, then 
	\[\bigoplus_{i\ge 0}\Ext_A^i(_Ak, _Ak)\]
	is Noetherian.
\end{conj}

\begin{conj}[Brown-Goodearl `98]
	If $A$ is a Noetherian Hopf algebra, then the injective dimension of $A_A$ is finite.
\end{conj}

These are both still open! In fact there is a meeting at Oberwolfach this March concerning
exactly these conjectures.

\subsection{Symmetric Tensor Categories}
We are going to be using the following notation throughout:
\begin{itemize}
	\item $k$ is a field
	\item $\Vectk$ is the category of $k$-vector spaces
	\begin{itemize}
		\item $\Vectk$ is closed under tensor products
		\item There is an element $k\in\Vectk$ such that
		\[k\otimes_k V\cong V\cong V\otimes_k k\]
		where the above isomorphisms are natural.
		\item $V\otimes_k W\cong W\otimes_k V$
	\end{itemize}
	\item An algebra is an object in $\Vectk$.
\end{itemize}

\begin{defn}\label{def-alg}
	$V\in\Vectk$ is called an \textbf{algebra object} if there are two morphisms
	\begin{enumerate}
		\item $m:V\otimes V\to V$
		\item $u:k\to V$
	\end{enumerate}
	such that the diagrams in figure \ref{fig-alg} commute.
	
\end{defn}
\begin{figure}\label{fig-alg}
	\begin{tikzcd}
		V\otimes V\otimes V\ar[r,"\id_V\otimes m"]\ar[d,"m\otimes \id_V"] & V\otimes V\ar[d,"m"]\\
		V\otimes V \ar[r,"m"] & V
	\end{tikzcd}
	\begin{tikzcd}
		k\otimes V\ar[r,"\sim"]\ar[rd,swap,"u\otimes\id_V"] & V & V\otimes k\ar[l,"\sim",swap]\ar[dl, "\id_V\otimes u"]\\
		& V\otimes V \ar[u,"m"] &
	\end{tikzcd}
	\caption{Diagrams for definition \ref{def-alg}.}
\end{figure}

\begin{lem}
	$V\in\Vectk$ is an algebra object iff $V$ is an algebra over $k$.
\end{lem}

\begin{lem}
	If $C$ is a symmetric tensor category, so is $C^{op}$.
\end{lem}

Then the natural thing to ask is: what is an algebra object in this opposite category?
\begin{defn}
	A \textbf{coalgebra object} in $C$ is an algebra object in $C^{op}$. Here we have 
	comultiplication $\Delta$ and counit $\varepsilon$.
\end{defn}
\begin{rmk}
	Naturally you could go about defining this from first principles and drawing the diagrams
	in figure \ref{fig-alg} with the arrows reversed, but we are probably mature enough
	to do without that (saving my fingers from repetitive strain injury in the process.)
\end{rmk}

\begin{lem}
	$\Algk$, defined as the category of algebra objects in $\Vectk$, is a symmetric tensor category. Furthermore
	$\Coalgk$, the category of coalgebra objects in $\Vectk$, is a symmetric tensor category.
\end{lem}

\begin{lem}
	The following are equivalent:
	\begin{enumerate}
		\item $V$ is an algebra object in $\Coalgk$
		\item $V$ is a coalgebra object in $\Algk$
		\item There are morphisms $m,u,\Delta,\varepsilon$ such that
		\begin{itemize}
			\item $(V,m,u)$ is an algebra
			\item $(V,\Delta,\varepsilon)$ is a coalgebra
			\item Equivalently:
			\begin{itemize}
				\item $m$ and $u$ are coalgebra morphisms
				\item $\Delta$ and $\varepsilon$ are algebra morphisms.
			\end{itemize}
		\end{itemize}
	\end{enumerate}
\end{lem}
\begin{prf}
	The nice thing here is that the $(a)\Leftrightarrow (c)$ without the last condition. A similar fact holds
	for $(b)$ except the second-to-last. The last thing to do is to prove the last two conditions are equivalent.
\end{prf}
\begin{prob}
	Fill in the details for the proof above.
\end{prob}
\begin{sol}
	Assume that $(V,m,u,\Delta,\varepsilon)$ is an algebra and coalgebra and further that
	$m$ and $u$ are coalgebra morphisms. That means in particular that the diagrams in figure \ref{fig-coalg-mor} commute.
	\begin{figure}[h]\label{fig-coalg-mor}
		\centering
		\begin{tikzcd}
			V\otimes V \ar[d,swap,"\Delta\otimes\Delta"]\ar[r,"m"] & V\ar[r,"\Delta"] & V\otimes V\\
			V^{\otimes 4}\ar[rr,"T_{2,3}"] & & V^{\otimes 4}\ar[u,swap,"m\otimes m"]
		\end{tikzcd}
		\begin{tikzcd}
			k\ar[r,"\Delta"]\ar[d,"u"] & k\otimes k\ar[d,"u\otimes u"]\\
			V\ar[r,"\Delta"] & V\otimes V
		\end{tikzcd}
		\label{fig-coalg-mor}
		\caption{$m$ and $u$ are coalgebra morphisms.}
	\end{figure}

	We are looking to prove that $\Delta$ and $\varepsilon$ are algebra morphisms, or that 
	the diagrams in figure \ref{fig-coalg-mor}  commute.
	\begin{figure}[h]
		\centering
		\begin{tikzcd}
			V\otimes V\ar[r,"m"]\ar[d,"\Delta\otimes \Delta"] & V\ar[d,"\Delta"]\\
			V^{\otimes 4}\ar[r,"m"] & V\otimes V
		\end{tikzcd}
		\begin{tikzcd}
			V\otimes V\ar[r,"\varepsilon\otimes\varepsilon"]\ar[d,"m"] & k\otimes k\ar[d,"m"]\\
			V\ar[r,"\varepsilon"] & k
		\end{tikzcd}
		\label{fig-alg-mor}
		\caption{$\Delta$ and $\varepsilon$ are algebra morphisms.}
	\end{figure}

	From here it's actually a bit boring because it's kinda just a definition/notation game.
	It boils down to the fact that the (co)multiplication on $V\otimes V$ has a twist that 
	exactly lines up so that each square is saying the same thing.
\end{sol}

\begin{defn}
	$V$ is called a \textbf{bialgebra object} if $V$ is an algebra object in $\Coalgk$.
\end{defn}

\begin{prob}
	\begin{enumerate}
		\item Suppose that $\ch k\ne 2$. Classify all bialgebras of $\dim 2.$
		\item Do the same for $\ch k = 2.$
	\end{enumerate}
\end{prob}
\begin{sol}
	\subsubsection*{Part (a)}
	Consider $\varepsilon:V\to k$ and consider $\ker\varepsilon\lhd V$. By rank-nullity, $\dim\ker\varepsilon= 1$, so
	$\ker\varepsilon=kx$ for some $x\in V$. Therefore $x^2=cx$ for some $c$, and if $c=0$, then (as an algebra) $V\cong k[x]/(x^2)$.
	Otherwise consider $y=\frac{x}{c}$. In this case $y^2=\frac{x^2}{c^2}=\frac{x}{c}=y$, and $V\cong k[x]/(x^2-x)$.

	Notice that in either case $\varepsilon(x)=0$, so let
	\[\Delta(x)=a(1\otimes 1)+b(1\otimes x)+c(x\otimes 1) + d(x\otimes x)\]
	and using that $\varepsilon\otimes \id\circ\Delta=\id\otimes\varepsilon\circ\Delta$
	and that each should be (essentially) the identity (this is just the diagram we saw before),
	we get $a=0$ and $b=c=1$. Thus the coalgebra structure of any Hopf algebra is given by 
	\[\varepsilon(x)=0,\quad \Delta(x)=1\otimes x+x\otimes 1+d(x\otimes x).\]

	Consider first the case when $x^2=0$. Then since comultiplication will be an algebra morphism,
	\[0=\Delta(x^2)=\Delta(x)^2=1\otimes x^2+x^2\otimes 1+d^2(x^2\otimes x^2)+2(x\otimes x)+2d(x\otimes x^2)+2d(x^2\otimes x)\]
	and since $x^2=0$, 
	\[0=2(x\otimes x).\]
	But $x\otimes x$ is a basis element of $V\otimes V$, so $V$ can only have this algebra structure
	when $\ch k=2$. We will return to this in the next part.

	So then $x^2=x$ and using the computation above,
	\[1\otimes x+x\otimes 1+d(x\otimes x)=\Delta(x)=\Delta(x^2)=1\otimes x+x\otimes 1+(d^2+4d+2)(x\otimes x)\]
	so
	\[(d^2+3d+2)(x\otimes x)=0\quad\Rightarrow\quad d^2+3d+2=(d+2)(d+1)=0\]
	and so either $d=-1$ or $d=-2$.

	One can verify that $\Delta$ is coassociative, so we can conclude that when $\ch k\ne 2$, 
	there are precisely two Hopf algebra structures with algebra structure $k[x]/(x^2-x)$ and
	comultiplication either
	\[\Delta(x)=1\otimes x+x\otimes 1-x\otimes x\quad\text{or}\quad \Delta(x)=1\otimes x+x\otimes 1-2(x\otimes x)\]

	\subsection*{(b)}
	Now assume that $\ch k=2$ and that $V\cong k[x]/(x^2-x)$ as an algebra. Then using the analysis above, 
	we see that we can choose comultiplication either
	\[\Delta(x)=x\otimes 1+1\otimes x\quad\text{or}\quad \Delta(x)=1\otimes x+x\otimes 1+x\otimes x.\]

	If instead $V\cong k[x]/(x^2)$, then \textit{any} value of $d$ will suffice, so there are a full $k$'s worth
	of Hopf algebra structures that can appear.
\end{sol} 

\section{January 9, 2019}
Today we are going to rely heavily on Sweedler notation. :) Notice that if we are looking at actual objects in the 
diagram for coassociativity, we get
\begin{figure}[h]
	\centering
	\begin{tikzcd}
		v\ar[r]\ar[d] & \sum v_{(1)}\otimes v_{(2)}\ar[d]\\
		\sum v_{(1)}\otimes v_{(2)}\ar[r] & \sum (v_{(1)})_{(1)}\otimes (v_{(1)})_{(2)}\otimes v_{(2)}=\sum v_{(1)}\otimes (v_{(2)})_{(1)}\otimes (v_{(2)})_{(2)}
	\end{tikzcd}
	\caption{Coassociativity on elements in Sweedler notation}
\end{figure}

\begin{ex}
	Let $G$ be a group and $kG$ be the group algebra. The algebra structure arises as normal
	where $g\cdot h$ comes from the structure on $G$. Then $\Delta(g)=g\otimes g$ and this extends linearly.

	But then if you consider $\Delta(\sum c_gg)$, notice that by the nature of tensors this is not unique!
	So we will just write
	\[\Delta\left(\sum c_gg\right)=\sum_Gc_g(g\otimes g)=\sum v_{(1)}\otimes v_{(2)}\]
\end{ex}

\subsection{Algebra structure on $V\otimes V$}
We said earlier on that $\Algk$ is a symmetric \textit{tensor} category. But how do we
define the multiplication on the tensor product?

Well it all comes from the twist! We define
\[m_{V\otimes W}=(m_V\otimes m_W)\circ(\id_V\otimes \tau_{2,3}\otimes \id_W)\]
where $\tau_{2,3}$ is the twist morphism.

More simply, $u_{V\otimes W}:k=k\otimes k\to V\otimes V$ simply defined by $u_V\otimes u_W$.

So then when we say that $\Delta$ is an algebra morphism, we are saying that for all $v,w\in V$
\[\Delta(vw)=\sum(vw)_{(1)}\otimes(vw)_{(2)}=(\sum v_{(1)}\otimes v_{(2)})(\sum w_{(1)}\otimes w_{(2)})=\sum v_{(1)}w_{(1)}\otimes v_{(2)}w_{(2)}\]

\subsection{Hopf Algebras}
Already to the good stuff!
\begin{defn}
	$V\in\Vectk$ is a \textbf{Hopf algebra} if $V$ is a bialgebra together with an \textbf{antipode}
	$S:V\to V$ satisfying
	\[(S,\id_V)\circ \Delta=\varepsilon=(\id_V,S)\circ\Delta\]
\end{defn}
\begin{conj}
	If $V\in\Vectk$ is a Noetherian Hopf algebra, then $S$ is bijective.
\end{conj}

\subsection{History and Motivation}
Hopf himself was a topologist, so this is the first context in which it arose. In the 1940's,
he began studying Hopf algebras over $\Z_2$ graded $k$ vector spaces. For instance, the 
cohomology ring of topological space $X$ with coefficients in $k$.

Later, in combinatorics, they ended popping up. Looking at rings of symmetric functions and other places
gave some interesting examples.

Then in group theory you can define a functor from groups to Hopf algebras by $F(G)=kG$
with the diagonal map. The antipode is just the inverse.

Then with Lie algebras, you can look at $\mathcal{U}(L)$, the universal enveloping algebra is a Hopf algebra.

Finally with algebraic groups (yay!) we take an algebraic group $G$ and consider the ring of functions on it, 
which is again a Hopf algebra.

\brk

Some ``cousins'' of Hopf algebras: quasi, weak, multiplier, ribbon, quasi-triangular, etc Hopf algebras. 
Each has slightly different base category or restrictions.

\section{January 11, 2019}
The plan for today is to talk about:
\begin{itemize}
	\item Convolution Algebras
	\item Antipodes
	\item Duality
	\item (Co-)Modules
\end{itemize}

\subsection{Convolution Algebras}
 Let $\scrT$ be a symmetric tensor category. We can usually think of $\scrT=\Vectk$, but 
 there is a problem since $\Vectk$ is equivalent to the category of Hopf algebras over $k$,
 while this is not generally true.

 We also need that $\scrT$ is $k$-linear (that is, enriched as a category over $k$). This
 means that $\Hom_\scrT (A,B)\in\Vectk$.

 \begin{thm}
	Let $\scrT$ be as above. Then $\Hom_\scrT(C,A)$ is an algebra and $\Hom_\scrT(-,-):(\Coalg_\scrT)^{op}\times\Alg_\scrT\to\Alg_k$
	is a functor.
 \end{thm}
 \begin{prf}
	Let $A$ be an algebra object in $\scrT$ and $C$ be a coalgebra object in $\scrT$. Then $1_{\Hom} :=u_A\circ\varepsilon_C:C\to 1_\scrT$ and define
	\[f\ast g := m_A(f\otimes g)\Delta_C:C\to A.\]

	Then using Lemma \ref{lem-func} and the fact that $A$ and $C$ are (co)algebra objects,
	we can see that the product $\ast$ satisfies the axioms required.

	Note that actually
 \end{prf}
 \begin{lem}
	\begin{enumerate}
		\item $1_{\Hom}\ast f = m_A(u\otimes 1)(1\otimes f)(\varepsilon\otimes 1)\Delta.$
		\item $f\ast 1_{\Hom}=m_A(1\otimes u)(f\otimes 1)(1\otimes \varepsilon)\Delta$
		\item $(f\ast g)\ast h=m_A(m_A\otimes 1)((f\otimes g)\otimes h)(\Delta_C\otimes 1)\Delta_C$
		\item $f\ast(g\ast h)=m_A(1\otimes m_A)(f\otimes(g\otimes h))(1\otimes \Delta)\Delta$
	\end{enumerate}
	\label{lem-func}
 \end{lem}
 
 \begin{prf}
	\subsubsection*{(a)}
	\begin{align*}
		1_{\Hom}\ast f&= m_A(1_{\Hom}\otimes f)\Delta_C\\
		&=m_A(u_A\circ\varepsilon\otimes f)\Delta_C\\
		&=m_A(u\otimes 1)(\varepsilon\otimes 1)(1\otimes f)\Delta
	\end{align*}
	\subsubsection*{(b)}
	Same as (a), essentially.
	\subsubsection*{(c) and (d)}
	\begin{align*}
		(f\ast g)\ast h&= m_A((f\ast g)\otimes h)\Delta\\
		&=m_A((m_A(f\otimes g)\Delta)\otimes h)\Delta\\
		&=m_A(m_A\otimes 1)((f\otimes g)\otimes h)(\Delta\otimes 1)\Delta
	\end{align*}
	and the other is analogous.
 \end{prf}

 \begin{defn}
	$V\in\scrT$ is a \textbf{Hopf algebra object} if:
	\begin{itemize}
		\item $V$ is a bialgebra object in $\scrT$ and
		\item There is a map $S:V\to V$ that is $(\id_V)^{-1}$ with respect to $\ast$.
	\end{itemize}
 \end{defn}
 \begin{rmk}
	Notice here that $\id_V\in\Hom_\scrT(V,V)$, the identity map in $\scrT$. We are \textit{not}
	taking about $1_{\Hom}=u\circ\varepsilon$.

	Also, we call $S$ an \textbf{antipode.}
 \end{rmk}

 \subsection{Duality}
 Notice that when $C=1_\scrT$ (that is the tensor identity), $\Hom_\scrT(1_\scrT,-):\Alg_\scrT\to\Algk$
 is a functor. Same for the dual from $\Coalg_\scrT$. This second one gives us a chance to talk about duality.

 \begin{lem}
	Let $\scrT$ be the category of finite dimensional vector spaces over $k$.
	Then $(-)^*:\scrT\to\scrT^{op}$ is an equivalence.
 \end{lem}
 This uses $(V\otimes W)^*=W^*\otimes V^*$.
 \begin{cor}
	$V$ is an algebra over $k$ $\Leftrightarrow$ $V^*$ is a coalgebra over $k$.
	And vice versa.
 \end{cor}

 Recall $S=(\id_V)^{-1}$. Thus
 \[S\ast \id_V=1_{\Hom}=\id_V\ast S\]
 The diagram we have here is
 \begin{figure}[h]
	 \centering
	 \begin{tikzcd}
		V\otimes V\ar[rr,"S\otimes \id_V"] & & V\otimes V\ar[d,"m_V"]\\
		V\ar[u,"\Delta"]\ar[d,"\Delta"]\ar[r,"\varepsilon_V"] & k \ar[r,"u_V"] & V\\
		V\otimes V\ar[rr,"\id_V\otimes S"] & & V\otimes V\ar[u,"m_V"]
	 \end{tikzcd}
 \end{figure}

 \subsection*{Modules/Comodules}
 \begin{defn}
	Let $A$ be an algebra object in $\scrT$. A \textbf{left $A$ module} is $M\in\scrT$
	with a morphism
	\[m_M:A\otimes M\to M\]
	such that the diagrams in Figure \ref{fig-mod} commute.
\end{defn}
\begin{figure}[h]\label{fig-mod}
	\centering
	\begin{tikzcd}
		A\otimes A\otimes M\ar[r,"m_A\otimes 1"]\ar[d,"1\otimes m_M"] & A\otimes M\ar[d,"m_M"]\\
		A\otimes M\ar[r,"m_M"] & M
	\end{tikzcd}
	\begin{tikzcd}
		A\otimes M\ar[r,"m_M"] & M\\
		1_\scrT\otimes M\ar[ur,"\sim"]\ar[u,"u_A\otimes 1"] &
	\end{tikzcd}
	\caption{Module diagrams}
\end{figure}
\begin{rmk}
	Note that we don't necessarily need that $M$ lie in $\scrT$. We could instead just 
	rely on an algebra homomorphism $\varphi:A\to \Hom_{\scrT}(M,M)$ and proceed as usual.
\end{rmk}

\section{January 14, 2019}
Today we are talking about Hopf modules and later in the week we will see the fundamental
theorem of Hopf modules as well as a neat result.

Here is the ``example of the day.''
\begin{ex}
	Consider $k[x]$ with maps $\Delta(x^n)=\sum_0^nx^i\otimes x^{n-i}$ and $\varepsilon(x^n)=\delta_{1,n}$. 
	$S(x^n)=(-x)^n$. Then notice that $\Delta(x)=1\otimes x+x\otimes 1$, so $\delta(x^n)=(\delta x)^n$.

	James doesn't want to do the rest of the computations, but they can be done. :)
\end{ex}

\begin{prob}\label{prob-4.1}
	(**) Working with a Hopf algebra $V$, consider the convolution algebra $\Hom(V\otimes V,V)$
	\begin{enumerate}
		\item $(S\circ m)\ast m=1_{\Hom(V\otimes V,V)}=m\ast(m\circ s\otimes s\circ\tau)$ where $s$ is the antipode in $V$.
		\item $(\Delta\circ S)\ast \Delta=1_{\Hom(V,V\otimes V)}=\Delta\ast((s\otimes s)\circ\tau\circ \Delta)$
	\end{enumerate}
\end{prob}
\begin{sol}
	\begin{align*}
		(s\circ m)\ast m
	\end{align*}
\end{sol}
\begin{prob}
	Classify all Frobenius (to be defined) Hopf algebras of dimension 3.
\end{prob}

\subsection{Returning to (co)modules}\label{subsec-comod}
\begin{lem}
	Let $\scrT=\Vectk$, and $V$ and algebra over $k$. The following are equivalent:
	\begin{enumerate}
		\item $M$ is a left $V$-module (see last lecture).
		\item There is an action of $A$ on $M$ such that $(ab)\cdot m=a\cdot (b\cdot m)$ and $1\cdot m=m$.
		\item There is an algebra morphism $\varphi:V\to\Hom_k(M,M)$
	\end{enumerate}
\end{lem}

\begin{defn}
	Let $V$ be a coalgebra over $k$. We say $M$ is a left \textbf{comodule} if there is a map
	\[\rho_M:M\to V\otimes M\]
	such that the diagrams in figure \ref{fig-com} commute.
\end{defn}
\begin{figure}[h]\label{fig-com}
	\begin{tikzcd}
		M\ar[r,"\rho"]\ar[d,"\rho"] & V\otimes M\ar[d,"\id_V\otimes\rho"]\\
		V\otimes M\ar[r,"\Delta_V\otimes\id_M"] &V\otimes V\otimes M
	\end{tikzcd}
	\begin{tikzcd}
		M\ar[r,"\rho"]\ar[d,"\cong"] & V\otimes M\ar[dl, "\varepsilon\otimes\id_M"]\\
		k\otimes M
	\end{tikzcd}	
	\caption{Diagrams for the definition of a comodule.}
\end{figure}

\begin{rmk}
	Notationally speaking, we write $_V\mathcal{M}$ and $^V\mathcal{M}$ for the categories of 
	left modules and comodules. Similar for right boyes.
\end{rmk}

\subsection{Tensors of $V$-modules}
Say $V$ is a bialgebra (or Hopf if you prefer, but it's not necessary). Then we get the following:
\begin{lem}
	Let $M$ and $N$ be two right $V$-modules. Then
	\begin{enumerate}
		\item $M\otimes N$ is a right $V$-module.
		\item If $V$ is cocommutative, then $M\otimes N\cong N\otimes M$ as right $V$-modules.
		\item $\mathcal{M}_V$ is a tensor (monoidal) category.
	\end{enumerate}
\end{lem}
\begin{prf}
	For (a), use the fact that $V\otimes V$ acts on $M\otimes N$ in a natural way. Then you get
	a map $V\otimes V\to \Hom_k(M\otimes N,M\otimes N)^{op}$ (why opposite?) and by precomposing
	with $\Delta$ to show they are $V$-modules. This establishes (b).

	Finally (c) follows because $\Vectk$ is a tensor category.
\end{prf}

\subsection{Hopf Modules}\label{subsec-hopfmod}
\begin{defn}
	Let $V$ be a bialgebra over $k$. We say that $M$ is a $\binom{r}{r}$ Hopf $V$-module if
	\begin{enumerate}
		\item $(M,m_M)$ is a right $V$-module
		\item $(M,\rho_M)$ is a right $V$ comodule.
		\item $\rho_M$ is a right $V$-module map.
		\begin{itemize}
			\item Equivalently, $m_M$ is a right $V$-comodule map.
		\end{itemize}
	\end{enumerate}
\end{defn}
\begin{rmk}
	The fact that these two last conditions are equivalent are perhaps not immediately obvious but we
	have been assured it is true. :)
\end{rmk}

\section{January 16, 2019}
\begin{thm}[Larson-Sweedler `69]\label{thm-LS69}
	Let $V$ be a Hopf algebra over $k$. Then the following categories are equivalent:
	\[_V^V\mathcal{M}\cong \Vectk\cong \mathcal{M}_V^V\]
\end{thm}

\begin{rmk}
	A natural question that one may ask is how to extend this theorem to the enriched setting.
\end{rmk}

\begin{lem}
	$S$ is an anti-homomorphism of an algebra $V$ and an anti-homomorphism of the coalgebra $V$.
\end{lem}
\begin{prf}
	Use the facts proved in problem~\ref{prob-4.1}.
\end{prf}

\begin{prob}
	If $\ch k=0$ any Hopf quotient of $k[x]$ is either $k$ or $k[x]$ itself.
\end{prob}

\begin{prob}
	Suppose $k$ has positive characteristic. Construct two different Hopf algebra quotients of $k[x]$ of dimension $p=\ch k$.
\end{prob}

\begin{ex}
	$\ch k=p>0$. Let $V=k[x]/(x^p)$ as an algebra and define $\Delta(x)=x\otimes 1+1\otimes x$,
	$\varepsilon(x)=0$ and $S(x)=x$. Then notice
	\[\Delta(x^p)=\sum\binom{p}{i}x^i\otimes x^{p-i}=x^p\otimes 1+1\otimes x^p.\]
\end{ex}

\subsection{Hopf Modules Once More}
Recall the definitions from section~\ref{subsec-comod} and section~\ref{subsec-hopfmod}.

\begin{rmk}
	An equivalent (more category-theoretical) definition of a Hopf module is a $V$-comodule object in 
	the category of $V$-modules. Also can dualize everything.
\end{rmk}

\begin{ex}
	Consider $V\in _V^V\mathcal{M}$. THen you can define $\rho=\Delta$ and check it satisfies all the requirements.
\end{ex}
\begin{defn}
	An $\binom{l}{l}$ Hopf $V$-module $M$ is called \textbf{trivial} if $M\cong V\otimes M_0$ for some $M_0\in\Vectk.$
\end{defn}
\begin{rmk}
	Basically you can think of this being trivial since we can \textit{always} define a module in this way
	where the entire module structure is inherited from the structure of $V$ (that is, irrespective of $M_0$).
\end{rmk}
\begin{thm-prime}{thm-LS69}%[Fundamental Thm of Hopf Modules (L-S `69)]
	Suppose that $V$ is a Hopf algebra. Then
	\begin{enumerate}
		\item Every $\binom{l}{l}$ Hopf $V$-module is trivial.
		\item Let $M\in _V^V\mathcal{M}$. Then 
		\[M\xrightarrow{\sim} V\otimes M^{Cov}\]
		where
		\[M^{Cov}:=\{m\in M|\rho(m)=1\otimes m\}.\]
		\item $_V^V\mathcal{M}\cong \Vectk$.
	\end{enumerate}
\end{thm-prime}
\begin{rmk}
	This is actually just a reformulation of theorem~\ref{thm-LS69} in less compact (but ultimately more usable) notation.
\end{rmk}
\begin{prf}
	Some identites that will be useful:
	\begin{align*}
		\Delta(v)=\sum v_1\otimes v_2\\
		\sum S(v_1)\otimes v_2=\varepsilon(v)=\sum v_1S(v_2)\\
		\rho(m)=\sum m_{-1}\otimes m_1\\
		\sum v_1\otimes (v_2)_1\otimes (v_2)_2=\sum (v_1)_1\otimes (v_1)_2\otimes v_2=\sum v_1\otimes v_2\otimes v_3\\
		\sum (m_{-1})_1\otimes (m_{-1})_2\otimes m_0=\sum m_{-1}\otimes (m_0)_{-1}\otimes (m_0)_0=\sum m_{-2}\otimes m_{-1}\otimes m_0
	\end{align*}

	Then we defin $\phi:M\to M^{Cov}$ by 
	\[\phi(x)=\sum S(x_{-1})x_0\in M\]
	We claim first that $\phi(x)\in M^{Cov}$ -- namely $\rho(\phi(x))=1\otimes\phi(x)$. To see this, compute
	\begin{align*}
		\rho(\phi(x)) &= \sum (\phi(x))_{-1}\otimes (\phi(x))_0\\
		&= (S(x_{-1})x_0)_{-1}\otimes (S(x_{-1})x_0)\\
		&= \sum\Delta(S(x_{-1}))\rho(x_0)\\
		&= \sum (S\otimes S)\circ \tau(\Delta(x_{-2}))\cdot [x_{-1}\otimes x_0]\\
		&=\sum[S(x_{-2})\otimes S(x_{-3})][x_{-1}\otimes x_0]\\
		&= \text{my fingers are aching...}\\
		&= 1\otimes\phi(x)
	\end{align*}

	Now define $F:M\to V\otimes M^{Cov}$ by $F=(\id\otimes\phi)\circ\rho$ and $G:V\otimes M^{Cov}$ 
	to be the map taking $v\otimes m$ to $vm$. The next claim is that $GF$ and $FG$ are the identity.
	You can see this by similarly pushing around notation.
	
	Finally the last claim will be seen on Friday.
\end{prf}
\section{January 18, 2019}
Recall the Fundamental Theorem of Hopf Modules: $_V^V\mathcal{M}\simeq \Vectk$ or equivalently that
every (not necessarily finite dimensional) Hopf module is trivial: $M\cong V\otimes M^{Cov}$.

\section{Frobenius Algebras}
Today we will discuss the result that
\begin{thm}[Larson-Sweedler, `69]\label{LS-69-2}
	Every finite dimensional Hopf algebra is Frobenius.
\end{thm}

There are some nice categorical equivalences: finite dimensional Frobenius algebras over $k$ is
equivalent to 2-D quantum field theories, or equivalently the symmetric functor category from the 2-D cobordism category to $\Vectk$.

Whoa.

\begin{ex}
	$V=(kG)^*=\oplus_{g\in G}k\delta g$ where $G$ is a finite group. Define the algebra structure with
	\[1_V=\sum_{g\in G}\delta_g\]
	and $\delta_g\delta_h=\delta_g$ when $g=h$ and 0 otherwise.

	The coalgebra structure is $\varepsilon(\delta_g)=\delta_{g,1_G}$ where on the right it is the Kroneker delta.
	Furthermore $\Delta(\delta_g)=\sum_{h\in G}\delta_h\otimes\delta_{h^{-1}g}$ and $S(\delta_g)=\delta_{g^{-1}}$.
\end{ex}

\begin{prob}
	Prove that $kS_3\cong(kS_3)^*$.
\end{prob}
\begin{prob}
	Prove that $k\Z_3\cong(k\Z_3)^*$ when $\ch k\ne 3$.
\end{prob}

\begin{lem}
	Let $A$ and $B$ be algebras over $k$.
	\begin{enumerate}
		\item If $M\in_A\mathcal{M}_B$, $N\in \mathcal{M}_B$, then $\Hom_{\mathcal{M}_B}(_AM,N)\in\mathcal{M}_A$.
		\item Take $B=k$, $M\in _A\mathcal{M}$. Then
		\[M^*:=\Hom_k(M,k)\in\mathcal{M}_A.\]
		\item $(-)^*:_A\mathcal{M}_{f.d.}\to (\mathcal{M}_A)_{f.d.}^{op}$ is an equivalence of categories. This is called the \textbf{reflection principle.}
	\end{enumerate}
\end{lem}
\begin{prf}
	This is a known fact. (b) follows quickly from (a) and (c) involves extending the dual map to maps of left $A$-modules.
\end{prf}
By the principle of duality, we have:
\begin{lem}
	If $C$ and $d$ are finite dimensional coalgebras over $k$, then
	\begin{enumerate}
		\item if $M\in ^C\mathcal{M}^D_{f.d.}$ and $N\in \mathcal{M}^D_{f.d.}$ then $\Hom_{\mathcal{M}^D}(M,N)\in\mathcal{M}^C_{f.d.}$.
		\item If $M\in ^C\mathcal{M}_{f.d.}$ then $\Hom_k(M,k)\in\mathcal{M}^C$
		\item $(-)^*:^C\mathcal{M}_{f.d.}\to(\mathcal{M}_{f.d.}^C)^{op}$ is an equivalence of categories.
	\end{enumerate}
\end{lem}

\begin{defn}
	A finite dimensional algebra $A$ is called \textbf{Frobenius} if one of the following (equivalent)
	conditions hold:
	\begin{enumerate}
		\item $(A)^*_A\cong A_A$
		\item $_A(A)^*\cong _AA$
	\end{enumerate}
\end{defn}
\begin{rmk}
	Note that the equivalence of these two things follow due to the equivalence of categories we get
	from the two lemmas above.
\end{rmk}
We actually don't need the following lemma, but James is a fan so we're going to write it down.
\begin{lem}\label{lem-not-needed}
	Let $A$ be a finite dimensional algebra over $k$. Then $M$ is a finite dimensional left $A$-module if
	and only if $M$ is a finite dimensional right $A^*$-comodule.
\end{lem}
\begin{rmk}
	The reason we get the module to comodule switch is (partly?) due to the fact that in our
	monoidal category $(V\otimes W)^*=W^*\otimes V^*.$
\end{rmk}

\begin{lem}
	Let $V$ be a finite dimensional Hopf algebra and $M$ be a finite dimensional $\binom{l}{l}$ Hopf $V$-module.
	Then $M^*$ is a finite dimensional $\binom{r}{r}$ Hopf $V$-module.
\end{lem}
\begin{prf}
	$M\in {_V^V\mathcal{M}}$ if and only if $M\in_V\mathcal{M}\cap {^V\mathcal{M}}$ and the two structures are compatible.

	The fact that $M^*\in \mathcal{M}_V\cap\mathcal{M}^V$ is straightforward and checking the compatibility is easy to check.
\end{prf}

Now we finally prove that theorem~\ref{LS-69-2}:
\begin{prf}
	First of all $V\in{^V_V\mathcal{M}}$ so by the lemma $V^*\in\mathcal{M}_V^V$. By the fundamental theorem of Hopf modules,
	\[V^*\cong V\otimes(V^*)^{Cov}\cong V\]
	as Hopf modules. Thus $V_V^*\cong V_V$ in particular.
\end{prf}
\begin{rmk}
	The second isomorphism above comes from the fact that $(V^*)^{Cov}$ is one-dimensional, so is the trivial module. :)
\end{rmk}
\begin{cor}
	If $A$ is \textit{not} a Forbenius algebra, then there is \textit{no} Hopf algebra structure on $A$.
\end{cor}
\begin{ex}
	Some examples of algebras with(out) this property.
	\begin{enumerate}
		\item $\C[x]/(x^n)$ is finite dimensional and Frobenius, but if $n\ge 2$, there is \textbf{no Hopf structure.}
		\item $\C[x,y]/(x^2,y^2,xy)=\C1+\C x+\C y$. This has the property that it is local (unique maximal submodule over itself)
		but $A^*$ has two maximal submodules, so $A\not\cong A^*$, so not Frobenius.
	\end{enumerate}
\end{ex}
Finally we conclude with a proof of lemma~\ref{lem-not-needed}:
\begin{prf}
	Say that $\{a_i\}$ is a basis for $A$ and $\{a_i^*\}$ is a basis for $A^*$ ($a_i^*(a_j)=\delta_{ij}$).

	Define $\rho(m)=\sum(a_i\cdot m)\otimes a_i^*$. A lemma (use change of basis matrices) shows that
	$\sum a_i\otimes a_i^*$ is independent of choice of basis. Another lemma says (scalar) multiplication is associative
	if and only if $\rho$ is coassociative. This gets us the forward direction.

	For the reverse direction, say $\rho(m)=\sum m_0\otimes m_{-1}.$ then define $a\cdot m=\sum m_1(a)m_0$.
\end{prf}

\section{January 23, 2019}
Today we are interested in studying the representation category $_V\mathcal{M}$ for any Hopf algebra $V$. It will end up
(and this is a bit cryptic for now) that $k$ controls everything here.
\section{Represetnations and Modules}
Notice that for any Hopf algebra $V$ we have the exact sequence:
\[0\to \ker\varepsilon\to V\xrightarrow{\varepsilon} k\to 0\]
\begin{defn}
	The trivial $V$ modules is $V/\ker\varepsilon=k\in _V\mathcal{M}_V$.
\end{defn}
\begin{ex}
	$V=kG$ where $G$ is a group, $\varepsilon(g)=1$ for all $g\in G$. Then $\ker\varepsilon=\oplus k(g-1)$ and so
	\[k=V/\oplus k(g-1).\]
\end{ex}
\begin{ex}
	$V=k[x]$ where $\Delta(x)=1\otimes x+x\otimes 1$ and $\varepsilon(x)=0$. Then $\ker\varepsilon=\oplus_{n\ge 1}kx^n$, whence
	\[V/\ker\varepsilon=k[x]/(x).\]
\end{ex}
\begin{ex}
	Let $V=(kG)^*=\oplus_G k\delta_g$ and $\varepsilon(\delta_g)=\delta_{1,1_G}$. THen $\ker\varepsilon=\oplus_{g\ne 1_G}k\delta_g$ and
	\[V/\ker\varepsilon=k\delta_1\]
\end{ex}
The following uses $T_4$, the example of the day I missed due to my meeting.
\begin{ex}
	$T_4$: $\varepsilon(g)=1$, $\varepsilon(p)=0$. Then $\ker\varepsilon=k(g-1)\oplus kp\oplus kpg$ and
	\[V/\ker\varepsilon=k.\]
\end{ex}
\begin{lem}
	\begin{itemize}
		\item For all $x \in V$ and $a\in k$ (the trivial module), $xa=\varepsilon(x)a$.
		\item $k$ is the identity object in $_V\mathcal{M}$.
		\item If $V$ is cocommutative, then $_V\mathcal{M}$ is a symmetric tensor category.
	\end{itemize}
\end{lem}
\begin{prf}
	\[x\cdot a=(x-\varepsilon(x))a+\varepsilon(x)a=\varepsilon(x)a\]
	since $x-\varepsilon(x)\in\ker\varepsilon$.

	For the next part, consider that for every $M\in _V\mathcal{M}$, we have 
	\[M\otimes k\cong M\cong k\otimes M.\]
	But then for all $x\in V$, we have
	\begin{align*}
		\varphi[x\dot(m\otimes 1)]&=\varphi(x_1\cdot m\otimes x_2\cdot 1)\\
		&=\varphi(x_1\cdot m\otimes \varepsilon(x_2)\cdot 1)\\
		&=\varphi((x_1\varepsilon(x_2))m\otimes 1)\\
		&=\varphi(xm\otimes 1)=xm=x\varphi(m\otimes 1).
	\end{align*}
\end{prf}
\subsection{Integrals}
\begin{defn}
	An element $x\in V$ is called a \textbf{left integral} if $vx=\varepsilon(v)x$ for all $v\in V$.
	Similar for \textbf{right integral.}
\end{defn}
\begin{lem}
	The following are equivalent:
	\begin{itemize}
		\item $x\in V$ is a left integral.
		\item $kx\cong$ the trivial module.
		\item $1\mapsto x$ defines a left $V$-module morphism $k\to kx$.
	\end{itemize}
\end{lem}
\begin{lem}
	Let $\int_V^l$ denote the set of left integrals in $V$.
	\begin{itemize}
		\item $\int_V^l$ forms a vector subspace of $V$.
		\item $\int_V^l$ forms a left $V$ submodule of $V$. Thus it forms a left ideal of $V$.
		\item $\int_V^l$ forms a right ideal of $V$.
		\item $\int_V^l\cong\Hom_V(k,V)$ (left $V$-mods)
	\end{itemize}
\end{lem}
\begin{prf}
	The first three are relatively obvious. The last one comes from the map $\Phi:x\in\int_V^l\mapsto f_x:1\to x.$
\end{prf}
\begin{rmk}
	Unfortunately it ends up that the collection above ends up usually being zero. 
\end{rmk}
\begin{thm}
	Let $V$ be a finite dimensional Hopf algebra. Then $\dim\int_V^l=1$.
\end{thm}
\begin{prf}
	\begin{align*}
		\dim\int_V^l&= \dim\Hom_V(_Vk,_VV)\\
		&=\dim\Hom_V((V^*)_V,k_V^*)\\
		&=\dim\Hom_V(V_V,k)=\dim k=1
	\end{align*}
	Where we used above that $V$ was finite dimensional whence Frobenius to get $V^*\cong V$.
\end{prf}
\begin{ex}
	$V=kG$ for $G$ a finite group. Then $\int_V^l=\sum_G g$ and
	\[v\cdot(\sum_G g)=\varepsilon(v)(\sum_G g)=\int_V^r\]
	and when $v=h\in G$, we can compute it.
\end{ex}
\begin{ex}
	When $V=k[x]$, which is infinite dimensional, we get no integrals. This is because it is an 
	integral domain, so if
	\[x\cdot\int_V^l=\varepsilon(x)\int_V^l=0\cdot\int_V^l=0\]
	then this forces $\int_V^l=0$.
\end{ex}
\begin{ex}
	When $V=(kG)^*$ for $G$ a finite group, then 
	\[\int_V^l=\int_V^r=\delta_{1_G}\]
\end{ex}
\begin{ex}
	When $V=T_4=k1\oplus kg\oplus kp\oplus kgp$,
	\[\int_V^l=(g+1)p\ne\int_V^r=p(1+g)=(1-g)p\]
\end{ex}
\begin{ex}
	When $\ch k=p$ and $V=k[x]/(x^p)$, then 
	\[\int_V^l=\int_V^r=x^{p-1}.\]
\end{ex}

\section{January 28, 2019}
Today we will be mostly doing things in homological algebra. :)
\begin{ex}
	An example of an infinite dimension Hopf algebra is of the following: fix $q\in k^*$. Then 
	\[T_\infty=k<g, g^{-1}, p>/\langle gg^{-1}=g^{-1}g=1, gp=qpg\rangle\]
	and we can find $\{g^ip^j:i\in\Z, j\in\N\}$ is a $k$-linear basis.

	For the coalgebra structure, define $\Delta(g)=g\otimes g$ and $\varepsilon(g)=1$ and 
	$\Delta(p)=p\otimes 1+g\otimes p$ and $\varepsilon(p)=0$. Finally $S(g)=g^{-1}$ and $S(p)=q^{-1}pg^{-1}$.
\end{ex}

\subsection{Some nice results in Homological algebra}
\begin{prob}
	Prove that $\Q$ is an injective $\Z$-module.
\end{prob}
\begin{thm}[Auslander-Buchsbaum `59]
	Every local commutative algebra with finite global dimension is a UFD.
\end{thm}
Another result is about algebraic groups:
\begin{thm}
	If $\ch k=0$, then every Noetherian commutative Hopf algebra has finite global dimension
	and is a direct sum of integral domains, each of which is isomorphic.
\end{thm}
\begin{defn}[Semiprime]
	An ideal $I\lhd R$ is called \textbf{semiprime} if it is the intersection of (possibly infinitely many) primes.
\end{defn}
\begin{conj}
	Every Noetherian Hopf algebra of finite global dimension is semiprime (that is, 0 is semiprime in $V$).
\end{conj}

\begin{defn}[Connected Graded]
	An algebra $A$ is called \textbf{connected graded} if 
	\[A=k\oplus A_1\oplus A_2\oplus \cdots\]
	and $1_A=1_k\in k$, $A_iA_j\subseteq A_{i+j}$. 

	The new part (connected) refers to the fact that $k$ is a summand.
\end{defn}
\begin{thm}
	Suppose that $\ch k=0$. Let $V$ be a Noetherian Hopf algebra that is connected graded as an algebra (says nothing about the coalgebra structure).
	Then the following hold:
	\begin{itemize}
		\item $V$ has finite global dimension.
		\item $V$ is a domain.
		\item $V$ is Artin-Schelter regular.
		\item $V$ is Auslander regular and Cohen-Macaulay.
		\item $V$ is Calabi-Yau
		\item $V$ is an iterated Ore extension.
	\end{itemize}
\end{thm}
That's a lot of word salad.

\subsection{Projective Modules}
\begin{defn}[Module-Theoretic Definitions]
	Let $M$ be a (left) $A$ module. 
	\begin{enumerate}
		\item $M$ is \textbf{free} if $M\cong \oplus_\mathcal{I} A$
		\item $M$ is \textbf{projective} if it is a direct summand of a free module.
	\end{enumerate}
\end{defn}
\begin{lem}
	$p\in{_A\mathscr{M}}$ is projective if and only if the diagram in figure~\ref{fig-proj2} commutes.
\end{lem}
\begin{figure}[h]
	\centering
	\begin{tikzcd}
		& P\ar[d,"f"]\ar[dl,dashed,"F",swap] &\\
		M\ar[r,"\pi"] & N\ar[r] & 0
	\end{tikzcd}
	\caption{The definition of a projective object.}
	\label{fig-proj2}
\end{figure}
\begin{rmk}
	This actually is just the definition for any Abelian category. This may generalize even further, but this is enough for us. :)
\end{rmk}

\begin{defn}[Injective Module]
	An \textbf{injective} $V$-module is a projective $V^{op}$ module.
\end{defn}

\section{January 30, 2019}
Today we're talking more about homological algebra. In particular, we'll learn about (or see again)
\begin{itemize}
	\item Complexes
	\item Projective resolutions and dimension
\end{itemize}
First the example of the day:
\begin{ex}
	Consider $GL_2(k)$, a group -- in fact, an algebraic group! Then 
	\[\mathcal{O}(GL_2)=k[x_{11},x_{12},x_{21},x_{22},det^{-1}]/\langle det^{-1}(x_{11}x_{22}-x_{21}x_{12}-1\rangle\]
	Let $X=(x_{ij})\in GL_2(k)$.

	Define the coalgebra structure via $\Delta(X)=X\otimes X$ and $\Delta(x_{ij})=\sum_1^2 x_{is}\otimes x_{sj}$,
	$\varepsilon(X)=I_2$, $\varepsilon(x_{ij})=\delta_{ij}$. Some computation (this is not trivial!) give us
	\[\Delta(det)=det\otimes det\quad\Rightarrow\quad \Delta(det^{-1})=det^{-1}\otimes det^{-1}\]
	Then we can find that $S(X)=X^{-1}$ and $S(det^{\pm})=det^{\mp}$.
\end{ex}
\begin{prob}
	Construct $\mathcal{O}(GL_n)$.
\end{prob}
\begin{prob}
	Construct $\mathcal{O}(G)$.
\end{prob}
\subsection{Homological stuff}
\begin{defn}[Complex]
	A \textbf{complex} of $A$-modules is a sequence of $A$-modules connected by homomorphisms
	\[\cdots\to M_2\xrightarrow{d_2}M_1\xrightarrow{d_1} M_0\to \cdots\]
	where $d_i\circ d_{i+1}=0$ for all $i$.
\end{defn}
Then we went over the standard definitions for:
\begin{itemize}
	\item Homology
	\item (Short) Exact sequences
	\item Projective resolutions
	\item Projective dimension
\end{itemize}
\begin{defn}[Global (Homological) Dimension]
	The (left) \textbf{global dimension} is
	\[\gldim A=\max\{\projdim M| M\in {_A\mathcal{M}}\}\]
\end{defn}
\begin{thm}[Hilbert Syzygy Theorem]
	\[\gldim k[x_1,\dots,x_n]=n\]
\end{thm}
\begin{cor}
	\[\gldim k= 0\]
\end{cor}
\begin{defn}
	The \textbf{finitistic global dimension} of $A$ is defined to be
	\[\findim A=\max\{\projdim M| M\in{_A\mathcal{M}}, M\text{ is f.g.},\projdim M <\infty\}\]
\end{defn}
\begin{conj}[Finitistic Dimension Conjecture]
	Let $A$ be a finitely generated algebra. Then $\findim A<\infty$.
\end{conj}
\section{February 1, 2019}
A really cool theorem:
\begin{thm}[Lorenz-Lorenz `95]
	Let $V$ be a Hopf algebra. Then $\gldim V=\projdim {_Vk}$.
\end{thm}
\begin{prf}
	It is enough to show that $\gldim V\le\projdim{_Vk}$. Or equivalently that
	\[\projdim M\le \projdim k,\forall M\in{_V\calM}\]
	but then
	\[\projdim(M)=\projdim(k\otimes M)\le \projdim(k)\]
	by lemma~\ref{lem-tensorproj}
\end{prf}
\begin{ex}
	Today's example: Quantum group $V=\mathcal{O}_q(GL_2)$. Fix some $q\in k^*$. Then as an algebra, 
	\[V=k\langle x_{11},x_{12},x_{21},x_{22},det_q^{-1}\rangle/(\text{relations})\]
	where the relations are given by 
	\begin{align*}
		x_{ij}x_{kl}=(\delta_{ik}+\delta_{jl})x_{kl}x_{ij}\quad\text{except when } i=k \text{ and } j=l\\
		x_{12}x_{21}=x_{21}x_{12}\\
		x_{22}x_{11}-x_{11}x_{22}=(q-q^{-1})X_{12}x_{21}\\
		det_q^{-1}det_q=det_q det_q^{-1}=1
	\end{align*}
	where $det_q=x_{22}x_{11}-q^{-1}x_{12}x_{21}x_{11}x_{22}-qx_{12}x_{21}$. One can show $det_q$ is central in $V$.

	Now for the coalgebra: $\delta(X)=X\otimes X$, so $\Delta(x_{ij})=\sum_{s=1}^2 x_{is}\otimes x_{sj}$, $\varepsilon(X)=I_2$ and
	\[S(X)=X^{-1}=\begin{pmatrix}
		x_{22}det_q^{-1} & -qx_{12}det_q^{-1}\\
		-q^{-1}x_{21}det_q^{-1} & x_{11}det_q^{-1}
	\end{pmatrix}\]
\end{ex}
\begin{prob}
	Show that $\gldim(\begin{smallmatrix}\Z & \Q\\0 & \Q\end{smallmatrix})=2$ while $\gldim(\begin{smallmatrix}\Z & \Q\\0 & \Q\end{smallmatrix})^{op}=1$
\end{prob}

\subsection{Returning to Homological Algebra}
Let $A$ be an algebra, $M\in {_A\mathcal{M}}$. Recall the definitions of projective and global dimension.
\begin{rmk}
	The following are equivalent:
	\begin{enumerate}
		\item $\gldim A=0$
		\item $\gldim A^{op}=0$
		\item $A$ is semisimple Artininan
		\item $A=\oplus_1^b M_{n_i}(D_i)$ by Artin-Wedderburn where $D_i$ is a division ring.
	\end{enumerate}
\end{rmk}
\begin{rmk}
	\begin{enumerate}
		\item A PID has global dimension 1. 
		\item The free algebra $k\langle x_1,\dots,x_n\rangle$ has global dimension 1.
		\item The path algebra of a finite quiver has global dimension 1.
		\item $\mathcal{O}_q(GL_2)$ has global dimension 4.
	\end{enumerate}
\end{rmk}

From now on, let $V$ always stand for some Hopf algebra.
\begin{lem}
	${_V\mathcal{M}}$ is a tensor category. The action on $M\otimes N$ is the following:
	\[v\cdot (m\otimes n)=\sum (v_{(1)}\cdot m)\otimes (v_{(2)}\cdot n)\]
\end{lem}
\begin{lem}
	If $M\in {_V\mathcal{M}}$, then $V\otimes M\in {_V^V\mathcal{M}}$. As a corollary,
	\[V\otimes M\cong V\otimes (V\otimes M)^{cov}\]
	and as a consequence $V\otimes M$ is a free $V$-module.
\end{lem}

\begin{thm}
	Let $P\in {_V\calM}$ be projective. Then $P\otimes N\in {_V\calM}$ is projective for all $M\in{_V\calM}$
\end{thm}
\begin{prf}
	To see this, notice that $P$ is a summand of $V^{\oplus b}$. But then if $P\oplus Q\cong V^{b}$, then 
	\[[P\otimes N]\oplus[Q\otimes N]=[P\oplus Q]\otimes N=(V^b)\otimes N=(V\otimes N)^b\]
	and by the last lemma, this is free.
\end{prf}

\begin{lem}\label{lem-tensorproj}
	\[\projdim(M\otimes N)\le \min\{\projdim M,\projdim N\}\]
\end{lem}
\begin{prf}
	We only show that $\projdim M\otimes N\le \projdim M$. If $\projdim M=\infty$, we are done.
	If it is finite, take any minimal projective resolution of $M$ and tensor with $N$. This is a projective 
	resolution of $M\otimes N$ of length at most $\projdim M$. 
\end{prf}
\begin{rmk}
	The fact it is still exact holds because we can appeal to the vector space structure.
\end{rmk}

\brk

Some more results (mostly for finite dimensional algebras):
\begin{thm}[Radford `75]
	If $V$ is finite dimensional, then the antipode $S$ has finite order:
	\[S^d=\id_V\]
\end{thm}
\begin{thm}[Larson-Radford `88]\label{thm-LR88}
	Suppose $\ch k\ne 0$ (can be relaxed but gets more ugly). Then the following are equivalent:
	\begin{enumerate}
		\item $\gldim V=0$ ($V$ is semisimple Artinian)
		\item $\gldim V^*=0$
		\item $S^2=\id_V$
	\end{enumerate}
\end{thm}
\begin{thm}[Nichols-Zoeller `89]
	Let $V$ be finite dimensional. If $Q$ is a Hopf subalgebra of $V$, then $_WV$ and $V_W$ are free.
\end{thm}

\section{February 6, 2019}
First some history: in 1899, Maschke proves the following:
\begin{thm}
	Let $G$ be a finite group. Then $kG$ is semisimple (Artinian) if and only if $\ch k\nmid |G|$.
\end{thm}
Today, we will see an analog of Maschke for Hopf algebras:
\begin{thm}[``Maschke's Theorem'' (Larson-Sweedler `69)]
	Let $V$ be a finite-dimensional Hopf algebra. Then the following are equivalent:
	\begin{enumerate}
		\item $\gldim V = 0$ ($V$ is semisimple Artinian)
		\item $\varepsilon(\int^l)\ne 0$
		\item $\varepsilon(\int^r)\ne 0$
	\end{enumerate}
\end{thm}

Now say $V=kG$. Then $\int=\sum_{g\in G} g$ is both a left and right integral of $V$. This uses that $\varepsilon(g)=1$.

But then if $\varepsilon(\int)=\sum_G \varepsilon(g)=\sum_G 1=|G|\ne 0\Leftrightarrow \ch k\nmid |G|$.

\subsection{Example of the Day}
\begin{ex}[Kac-Paljutkin Algebra]
	$k=\C$ or any field with $\ch k\ne 2$. Then $V=k\langle x,y,z\rangle$ modulo the relations
	\begin{align*}
		x^2=y^2=1\\
		xy=yx\\
		xz=yz\\
		xy=xz\\
		x^2=\frac{1}{2}(1+x+y-xy)
	\end{align*}
	Then $V$ has a $k$-basis $\{1,x,y,z,xy,xz,yz,xyz\}$, so as an algebra $V\cong k^4\oplus M_2(k)$.

	Then $\Delta(x)=x\otimes x$, $\Delta(y)=y\otimes y$, $\Delta(z)=(1\otimes 1+1\otimes x+y\otimes 1-y\otimes x)(z\otimes z)$.

	And $\varepsilon(x)=\varepsilon(y)=\varepsilon(z)=1$ and $S(x)=x$, $S(y)=y$ and $S(z)=z$. Recall that although $S$ looks like the identity map, 
	$S$ is an \textbf{antihomomophism}. So actually $S^2=\id_V$.

	Claim:
	\[\int=1+x+y+z+xy+xz+yz+xyz\]
	then $\varepsilon(\int)=8\ne 0$, so $V$ is semisimple.

	Some nice facts:
	\begin{enumerate}
		\item $V^*\cong V$
		\item $V$ is the unique eight dimensional noncommutative, noncocommutative semisimple Hopf algebra.
		\item Let $Q_8$ be the quaternion group. Then $kQ_8$ is eight dimensional. Therefore $V\cong kQ_8$ as algebras, but \textbf{not as coalgebras.}
		The reason behind this is that $kG$ is \textit{always} cocommutative.
	\end{enumerate}
\end{ex}

\subsection{HW of the day}
\begin{prob}
	If $V$ is eight dimensional and noncommutative, then $V\cong k^4\oplus M_2(k)$ as an algebra. Thus if $G$ is nonabelian group of order 8, then $\C G\cong \C^4\oplus M_2(\C)$.
\end{prob}

\subsection{Proof of ``Maschke's theorem''}
Now we actually do the proof:
\begin{prf}
	(a) $\Rightarrow$ (b): Say the global dimension of $V$ is zero. Then $V$ is semisimple by Artin-Wedderburn and we have a decompositon
	\[V\cong \oplus_1^s M_{n_i}(D_i)\]
	and every minimal nonzero ideal has the form $M_{n_i}(D_i)$.

	Consider the ideal $k\int^l=I$ which is an ideal since $v\int^l=\varepsilon(v)\int^l\in I$. But then consider that if $v'\int^l\in I$, 
	\[v'\int^l v=\varepsilon(v')\int^l v\in k\int^l\]
	so in fact it is a two-sided ideal. Thus $k\int^l\cong M_{n_i}(D_i)$, so 
	\begin{align*}
		(k\int^l)^2&=k\int^l
		\varepsilon(\int^l)\int^l=(\int^l)^2&=a\int^l
	\end{align*}
	for some nonzero $a$. But then $\varepsilon(\int^l)\ne 0$.

	(b) $\Rightarrow$ (a): Now let $e=\frac{1}{\varepsilon(\int^l)}\int^l$. Then $e^2=e$ and $ke=k\int^l$ is a left $V$-module.
	But then $V=Ve\oplus V(1-e)$ so $ke=k\int^l$ is projective, so $\projdim_v k=0$.

	The last part follows by Lorenz-Lorenz -- $\gldim V=\projdim_V k$.
\end{prf}

\subsection{$\Ext$ groups}
We define these in the usual way through projective resolutions:

Let $A$ be an algebra. Then $_A\calM$ is an abelian cateogry, so $\Hom_A(M,N)\in\Vectk$.
\begin{defn}
	A (covariant) functor $F:{_A\calM}\to{_B\calM}$ is called \textbf{left exact} if, for any short exact sequence
	\[0\to L\to M\to N\to 0\]
	we have the exact sequence
	\[0\to F(L)\to F(M)\to F(N).\]

	Similar for right exact. If both, it's just exact.
\end{defn}
\begin{lem}
	For any $W\in{_A\calM}$, the functor $\Hom(W,-)$ is left exact.
\end{lem}
We also have that $\Hom_A(P,-)$ is exact iff $P$ is projective, so this gives us a nice characterization of projective objects in Abelian categories.
Then the category $\Ch({_A\calM})$ is the category of chain complexes in $A$ modules with chain maps.
\begin{lem}
	$\Ch({_A\calM})$ is an Abelian category.
\end{lem}
\begin{lem}
	A functor $F:{_A\calM}\to{_B\calM}$ can be extended to a functor between the appropriate chain complex categories.
\end{lem}

\subsection{Looking Forward}
On Friday, we may see the following result:
\begin{thm}
	Under some mild hypothesis (including finite dimensionality),
	\[\projdim_V k=\gldim V\]
\end{thm}

\section{February 13, 2019}
Here's the theorem for today (with the hypotheses in place):
\begin{thm}\label{thm-213}
	Assume $\ch k=0$ and $S^2=\id_V$. Then 
	\[\gldim V=\projdim{_VM}\]
	for all finite dimensional $V$ modules $M$.
\end{thm}
Another result is the theorem~\ref{thm-LR88}:
\begin{prf}
	Assuming theorem~\ref{thm-213}, (c)$\Rightarrow$(a) is obvious. That's only part of it, 
	obviously. :)
\end{prf}
\subsection{Example of the Day}
\begin{ex}
	$\mathcal{U}(\g)$, the universal enveloping algebra which (as an algebra) is isomorphic to
	$k\langle \g\rangle/(xy-yx-[xy])$. As a coalgebra, we define $\Delta(x)=x\otimes 1+1\otimes x$,
	$\varepsilon=0$ and $S(x)=-x$.

	This is a nice example for today, because notice that $S^2(x)=x$ and in fact $S^2=\id_{\mathcal{U}(\g)}$,
	so we get that $\gldim\mathcal{U}(\g)=\projdim_{\mathcal{U}(\g)}M$ for any finite dimensional $M$ (by 
	the theorem).
\end{ex}
\subsection{Today's Content}
\begin{lem}[Reflection Principle]\label{lem-ref]}
	Let $A$ be an algebra. Then
	\begin{enumerate}
		\item $(-)^*=\Hom)k(-,k):({_A\calM})_{fd}\to(\calM_A)_{fd}$ is a contravariant equivalence.
		\item $(M^*)^*\cong M$ for each $M\in{_AM}_{fd}$. In linear algebra, this says that we have a perfect pairing $M\otimes M^*\to k$.
	\end{enumerate}
\end{lem}
\begin{lem}[Restriction of Scalars]\label{lem-res1]}
	Let $f:A\to B$ be a map of algebras. Then
	\begin{enumerate}
		\item There is a functor $f^*:{_B\calM}\to{_A\calM}$; also 
		\item $f^*:\calM_B\to\calM_A$.
	\end{enumerate}
\end{lem}
\begin{lem}[Restriction of Scalars]\label{lem-res2}
	Let $f:A\to B$ be an antihomomophism of algebras. Then
	\begin{enumerate}
		\item There is a functor $f^*:{_B\calM}\to{\calM_A}$; also 
		\item $f^*:\calM_B\to{_A\calM}$.
	\end{enumerate}
\end{lem}
\begin{lem}
	Let $V$ be a Hopf algebra with antipode $S$. Then 
	\[\tilde S\circ (-)^*:{_V\calM}\to \calM_V\to {_V\calM}\]
	is a contravariant functor.
\end{lem}
\begin{cor}
	If $V$ is finite dimensional, then $\tilde S\circ(-)^*$ is a duality of $(_V\calM)_{fd}$.
\end{cor}
For the future we will use the notation $M^{*S}=\tilde S(M^*)$. Notice that for all $M\in{_V\calM},$ $M^{*S}\in{_V\calM}$.

This leads us to our proposition:
\begin{thm}
	$V$ is a Hopf algebra.
	\begin{enumerate}
		\item If $M\in{_V\calM}$ then there is a natural morphism
		\[ev_M:M^{*S}\otimes M\to {_Vk}\]
		\item if $M\in{_V\calM_{fd}}$, then there is a natural map
		\[coev_M:{_Vk}\to M\otimes M^{*S}\]
	\end{enumerate}
\end{thm}
\begin{prf}
	$ev_M(f\otimes m)=f(m)\in k$ is the map we're talking about. For all $v\in V$, 
	\begin{align*}
		ev_M(v\cdot(f\otimes m))&=ev_M(v_1f\otimes v_2 m)\\
		&=ev_M(fS(v_1)\otimes v_2m)\\
		fS(v_1)(v_2 m)\\
		&= f(S(v_1)v_2m)\\
		&= f(\varepsilon(v) m)=\varepsilon(v)f(m)=v\cdot ev_M(f\otimes m).
	\end{align*}

	If instead $M$ is finite dimensional, pick a $k$ basis $m_1,\dots,m_d$ for $M$ and let 
	$m_i^*$ be the dual basis. Set $coev_M(1)=\sum m_i\otimes m_i^*=\Phi$. Then use the fact that (under the identification of 
	$M\otimes M^*$ with $\Hom_k(M,M)$) we get $\Phi=\id_M$. Then prove 
	\[coev_M(v\cdot 1)=v\cdot coev_M(1).\]
\end{prf}
\begin{defn}
	The map $ev_M$ above is called the \textbf{evaluation map} and $coev_M$ is called the 
	\textbf{coevaluation map.}
\end{defn}

\begin{lem}
	\begin{enumerate}
		\item $(M^{*S})^{*S}\cong \tilde S^2(M)$ for all finite dimensional $M\in{_V\calM}$.
		\item If $S^2=\id$ then $(M^{*S})^{*S}\cong M$.
	\end{enumerate}
\end{lem}
\begin{lem}
	Let $M\in {_VM}_{fd}$ and suppose $S^2=1$. Then 
	\[ev_{M^{*S}}\circ coev_M=(\dim M)\id_k\]
\end{lem}

\section{February 15, 2019}

Here we use the (slightly extended) lemma:
\begin{lem}
	\begin{enumerate}
		\item $ev_M\circ coev_{M^{*S}}=(\dim M)\id_k$
		\item $\frac{1}{\dim M}ev_M\circ coev_{M^{*S}}=\id$
		\item (since the identity factors through $M^{*S}\otimes M$) $M^{*S}\otimes M\cong {_Vk}\oplus N$ as $V$-modules.
		\item $\projdim k\le \projdim M^{*S}\otimes M$.
	\end{enumerate}
\end{lem}
Then using this lemma, we can prove the theorem~\ref{thm-213}:
\begin{prf}
	Basically use this, Lorenz-Lorenz, and the result bounding $\projdim$ of a tensor module by its factors.
\end{prf}

Notice that if we change things up slightly we get a new theorem:
\begin{thm-prime}{thm-213}
	Let $V$ be a Hopf algebra. Suppose that $S^2=\id$. Then 
	\[\projdim_V M=\gldim V\]
	for all $M\in {_VM}_{fd}$ such that $\dim M\ne 0$ in $k$.
\end{thm-prime}

\begin{conj}
	Let $V$ be a Noetherian Hopf algebra over a field $k$ of characteristic zero and $S^2=\id$.
	Then
	\[\gldim V <\infty\]
\end{conj}

\begin{cor}
	If $V$ is a finite module over $\mathcal Z(V)$, its center, then the above conjecture holds.
\end{cor}

\subsection{Today's example}
He did the quantized enveloping algebra of $\sl_2$! Might be worth looking up.

An interesting part of this example is that $S^2$ is \textbf{not} the identity! In fact, 
this example is important in constructing Joon (sp?) polynomials in knot theory.

\subsection{Today's Homework}
\begin{prob}
	Prove the following ``dimensional shift'' lemma: Let
	\[0\to X\to P\to Y\to 0\]
	be a SES of $A$ modules, where $P$ is projective. If $M$ is an $A$ module, then for all $n\ge 1$:
	\[\Ext_A^{n+1}(Y,M)\cong\Ext^{n}_A(X,M).\]
\end{prob}
\begin{sol}
	Take a projective resolution for $X$ and try to attach it to the SES.
\end{sol}
\subsection{Classifying Algebras}
We have the (descending) chain of classes of algebras:
\begin{itemize}
	\item Algebras
	\item Noetherian Algebras
	\item Cohen-Macaulay Algebras
	\item Gorenstein Algebras
	\item Complete Intersections
	\item Hypersurface Rings
	\item Regular Algebras
\end{itemize}
The question we have is: where to (Noetherian) Hopf algebras lie in this hierarchy?

If we restrict to finite-dimensional Hopf algebras, Brown-Goodearl proved that Noetherian Hopf algebras
are all Gorenstein rings.

\begin{defn}
	A Noetherian algebra $A$ is called \textbf{regular} if $\gldim A<\infty$.
\end{defn}
\begin{rmk}
	Notice that since $A$ is Noetherian, we didn't have to talk about left vs. right global dimension.
\end{rmk}
\begin{ex}
	If $L$ is a finite dimensional Lie algebra, then $U(L)$ is regular:
	\[\gldim U(L)=\dim L.\]
\end{ex}
\begin{defn}
	An element $x\in A$ is called \textbf{normal} if $xA=Ax$ (NOT $xa=ax$)
\end{defn}
\begin{defn}
	A Noetherian algebra $A$ is called a \textbf{hypersurface ring} if there exists a Noetherian 
	regular algebra $B$ and a normal non-zero-divisor $x\in B$ such that $A\cong B/(x)$
\end{defn}
\begin{ex}
	$\ch k=p>0$, $V=k[x]/(x^p)$, a Hopf algebra with comultiplication $\Delta(x)=x\otimes 1+1\otimes x$.
	Then $V$ is a hypersurface ring.
\end{ex}
\begin{defn}
	A Noetherian algebra $A$ is called a \textbf{complete intersection} if there is a Noetherian regular algebra
	$B$ and a sequence of elements $x_1,\dots,x_d$ such that:
	\begin{itemize}
		\item $x_1$ is a non-zero-divisor in $B$.
		\item $x_i$ is a NZD in $B/(x_1,\dots,x_{i-1})$ for all $i$
		\item $A\cong B/(x_i)$.
	\end{itemize}
\end{defn}

\begin{defn}
	A Noetherian algebra $A$ is called \textbf{Gorenstein} if 
	\[\injdim{_AA}=\injdim A_A<\infty.\]
\end{defn}
\begin{defn}
	A Noetherian algebra $A$ is called \textbf{Cohen-Macaulay} if there is an $A$-bimodule
	$M$ such that
	\begin{itemize}
		\item $\injdim {_AM}<\infty$, $\injdim M_A<\infty$
		\item $\End(M_A)\cong A,$ $\End({_AM})\cong A^{op}$
		\item ${_AM}$ and $M_A$ are Noetherian.
	\end{itemize}
\end{defn}

\section{February 20, 2019}
Last time we wrote out a big chain of inclusions for algebras. We didn't really discuss why all complete
intersections are Gorenstein algebras. Actually, this is one of the harder inclusions to prove. We will at least 
try today, but will definitely get to it by Friday.

\subsection{Example of the day}
\begin{defn}
	$V=k[g^{\pm 1}][y;\delta]$, an Orr extension of the Laurent polynomial ring by the derivation $\delta=(g^n-g)\frac{d}{dg}$.
	For those (me) wo haven't seen Orr extensions before, we also have
	\[V\cong k\langle g, g^{-1}, y\rangle/(yg-gy-(g^n-g), gg^{-1}-1, g^{-1}g-1)\]

	This is a domain where $\gldim V=2$ (also the GK dimension, which we haven't defined). As a coalgebra, we get
	$g$ is grouplike and $\Delta y= y\otimes g^{n-1}+1\otimes y$, $\varepsilon(y)=0$, and $S(y)=-yg^{-(n+1)}$.

	Notice that $V$ is regular.
\end{defn}
\subsection{Problem of the day}
\begin{prob}
	Let $x$ be a normal non-zero-divisor in $A$ where $xA=Ax$. Then for all $a\in A$ let $\sigma\in\Aut A$ be defined such that
	$\sigma(a) x= xa$.

	Let $M\in {_A\calM}$ and define a left $A$ action on ${^\sigma M}$ whose underlying set is $M$ and the action is defined by $a\cdot m=\sigma(a)m$. In terms of the action (morphism), this is 
	equivalent to precomposing with $\sigma:$ 
	\[A\xrightarrow{\sigma} A\xrightarrow{\rho}\End(M)\]

	Prove the following lemma:
	\begin{lem}[Rees Lemma]
		Let $x$ be a normal non-zero-divisor in $A$. If $L$ is a left $A/(x)$-module and $M$ is an $x$-torsion-free $A$-module, then
		\[\Ext_{A/(x)}^n(L,M/xM)\cong \Ext_A^{n+1}(L,{^\sigma M}).\]
	\end{lem}
\end{prob}

\subsection{Today's Results}
We begin with some corollaries of Rees Lemma:
\begin{cor}
	\begin{enumerate}
		\item $\Ext_{A/(x)}^n(L, A/(x))\cong \Ext_A^{n+1}(L,{^\sigma A})\cong\Ext_A^{n+1}(L,A)$
		\item $\injdim_{A/(x)}(A/(x))\le \injdim_AA-1$
		\item If $A$ is Gorenstein, so is $A/(x)$.
	\end{enumerate}
\end{cor}
\begin{rmk}
	Then the idea here is that we just have to prove that regular rings are Gorenstein, since then complete intersections
	are just (finite) sequences of quotients by regular elements.
\end{rmk}
\subsection{Doing $\Ext$ again}
Let $P^\bullet$ and $Q^\bullet$ be complexes of $A$-modules. Then $\Hom_A(P^\bullet, Q^\bullet)$ is a complex. Define
\[\Hom_A(P^\bullet, Q^\bullet):=\prod_{i\in\Z}\Hom_A(P^i,Q^{i+n}=\prod_{-i+j=n}\Hom(P^i,Q^j)\]
where we define the boundary map to be 
\[d_{\Hom_A(P^\bullet,Q^\bullet)}:(f_i)_{i\in\Z}\mapsto (d_Q^{n+1}\circ f^i - (-1)^nf^{i+1}\circ d_P^i)_{i\in\Z}\]
\begin{prob}
	The \textbf{real} homework for this class (the other one is what we're showing, apparently) is to show that this is a
	chain complex -- i.e. that $d^2=0$.
\end{prob}

Define the projective resolution to be the regular projective resolution, but do not include $M$ itself. But then we have
\begin{align*}
	R\underline{\Hom}=\Ext_A^n(M,N)&:=H^n(\Hom_A(P^\bullet_M,N))\\
	&:=H^n(\Hom_A(P_M^\bullet,P_N^\bullet))\\
	&:=H^n(\Hom_A(I_M^\bullet, I_N^\bullet))\\
	&:=H^n(\Hom_A(M,I_N^\bullet))
\end{align*}

\subsection{Proof plz}
Today we will try to get $n=0$ in the inductive argument.
\begin{lem}
	\begin{enumerate}
		\item  There is a short exact sequence of left $A$ modules
		\[0\to {^\sigma A}\xrightarrow{l_x}A\to A/(x)\to 0\]
		where $l_x$ sis left multiplication by $x$.
		\item If $M$ is $x$-torsion-free, then there is a SES of left $A$ modules
		\[0\to {^\sigma M}\xrightarrow{l_x}M\to M/xM\to 0\]
	\end{enumerate}
\end{lem}
The second item implies the first and neither is hard to show directly.
\begin{lem}
	Let $M$ be an $x$-torsion-free left $A$ modules and let $I_M$ be an injective resolution of $M$. THen there is a morphism of complexes
	$l_x:{^\sigma I_M}\to I_m$.
\end{lem}
Again, not too hard to check.

\brk

Notice now that $xL=0$ by the hypotheses. Combining this with the last result, we have
\begin{lem}
	\begin{enumerate}
		\item For any $L\in {_A\calM}$,
		\[\Ext_A^n(L,l_x):\Ext_A^n(L,{^\sigma M})\to \Ext_A^n(L,M)\]
		\item If $xL=0$ then the above map is zero for all $n\ge 0$.
	\end{enumerate}
\end{lem}
Again, follow your nose a bit. $f:L\to {^\sigma I^i}$ is an $A$ homomorphism and pull that $x$ in.

Applying $\Hom_A(L,-)$ to $0\to {^\sigma M}\xrightarrow{l_x}M\to M/xM\to 0$, we obtain the long exact sequence
for $\Ext$. But then notice that $\Hom_A(L,M)=\Ext_A^1(L,M)=0$ since $M$ is $x$-torsion-free (and $L$ is killed by $x$).
and since $\Ext_A^1(L,{^\sigma M})\cong \Hom_A(L,M/xM)\cong\Hom_{A/(x)}(L,M/xM)$, we get the base case proven.

\section{February 22, 2019}
Today we are going to finish the proof of Rees, then give some remarks, then give the example of the day. :)

\subsubsection{Continuation of Proof of Rees}
Last time we proved that $\Ext_{A}^1(L,{^\sigma M})\cong \Ext_{A/(x)}^0(L,M/xM)$. Now we proceed by induction.

Consider the SES $0\to K\to F\to L\to 0$ of left $A/(x)$-modules where $F$ is free. Then we get a long exact sequence
\begin{align*}
	0\to \Hom_{A/(x)}(L,M/xM)\to \Hom_{A/(x)}&(F,M/xM)\to \Hom_{A/(x)}(K,M/xM)\\
	&\to \Ext^1_{A/(x)}(L,M/xM)\to\Ext^1_{A/(x)}(F,M/xM)\to\cdots
\end{align*}
as well as the sequence
\begin{align*}
	\cdots \to \Ext^1_{A}(L,{^\sigma M})\to \Ext^1_{A}&(F,{^\sigma M})\to \Ext^1_{A}(K,{^\sigma M})\\
	&\to \Ext^2_{A}(L,{^\sigma M})\to \Ext^2_{A}(F,{^\sigma M}\to\cdots)
\end{align*}
and we have isomorphisms between all the corresponding parts (except the fourth term). THe first three 
follow from $n=0$ and the last one follows since $F$ is a projective $A/(x)$ module and since $F$ has projective dimension (over $A$) 
less than or equal to one. Then five lemma proves that this holds for $n=1$.

Then you can just extend both sequences and iteratively prove this holds for each term. Woot. :)

\subsection{Return to Classifications}
Recall we have the (descending) category chain
\begin{itemize}
	\item Algebras
	\item Noetherian Algebras
	\item Cohen-Macaulay Algebras
	\item Gorenstein Algebras
	\item Complete Intersections
	\item Hypersurface Rings
	\item Regular Algebras
\end{itemize}
Now we can show by applying Rees directly:
\begin{cor}
	$\injdim A/(x)\le \injdim A$
\end{cor}
Which then allows us to prove:
\begin{cor}
	Every complete intersection ring is Gorenstein.
\end{cor}
\begin{prf}
	Let $A$ be a CI ring. Then there exists a Noetherian regular ring $B$ and a sequence of elements $b_1,\dots, b_n\in B$ 
	such that each $b_i$ is a normal nonzerodivisor on $B/(b_1,\dots,b_{i-1})$.

	Since $B$ is regular, it is Gorenstein. By the corollary $B/(b_1)$ is Gorenstein. Continuing this way, $A$
	itself is Gorenstein and we are done. 
\end{prf}

\subsection{Where do Hopf algebras fit in?}
Recall that we proved that (finite dimensional) Hopf algebras are Frobenius algebras. Well it ends up
that Frobenius algebras are also Gorenstein! Thus all (f.d) Hopf algebra is Gorenstein.

Recall the Brown-Goodearl conjecture: all Noetherian Hopf algebras are Gorenstein. This is still an open conjecture!
But it gets worse!
\begin{conj}
	Every Noetherian Hopf algebra is a complete intersection.
\end{conj}
\begin{conj}
	Every finite dimensional Hopf algebra is a complete intersection.
\end{conj}
There are no known counterexamples to either! This has been open for decades with no satisfactory result.
\subsubsection{Even more classes of algebras!}
Inside the category of Gorenstein rings, there are a couple more classes (not related to complete intersections):
We get the decreasing chain
\begin{itemize}
	\item Gorenstein rings
	\item Artin-Schelter Gorenstein
	\item A-S Gorenstein with skew Calabi-Yau property
\end{itemize}
and over at regular algebras we get the decreasing chain
\begin{itemize}
	\item Regular algebras
	\item A-S regular algebras
	\item Skew C-Y algebras
	\item C-Y algebras
\end{itemize}
\begin{defn}
	A Noetherian Gorenstein ring $A$ of $\injdim$ d is called $A-S$ Gorenstein if, for all $S\in{_A\calM}_{fd},$
	\[\Ext_A^i(S,A)=\left\{\begin{array}{lr}0 & i\ne d,\\ \in {_A\cal M}_{fd}, & i=d\end{array}\right.\]
	where the above property is called the Artin-Schelter (AS) property.
\end{defn}
\begin{defn}
	A Noetherian A-S Gorenstein algebra $A$ of injective dimension $d$ is called \textbf{skew Calabi-Yau} if
	\[\Ext_{A^e}^i(A,A^e)=\left\{\begin{array}{lr}0 & i\ne d,\\ {^\sigma A}, & i=d\end{array}\right.\]
	(remember that $A^e\cong A\otimes A^{op}$).
\end{defn}
\begin{defn}
	$A$ is Calabi-Yau if it is skew C-Y and also $\Ext_{A^e}^d(A,A^e)=A$.
\end{defn}
\begin{rmk}
	The idea here (and honestly I missed some parts of the discussion) is that C-Y algebras are ones where we can get 
	Poincar\'e duality.
\end{rmk}
\subsection{Example of the Day}
\begin{ex}
	$V$ is generated by $g,g^{-1},x_2,x_3$ subject to
	\[gg^{-1}=g^{-1}g=1,\quad gx_i=\eta^ix_i g,\quad x_2^3=x_3^2\quad x_2x_3=x_3x_2\]
	where $\eta$ is a primitive $6^{th}$ root of unity.

	Then $\gldim V=\infty$ and it is a kypersurface ring, since
	\[V\cong k[x_1,x_2][g^{\pm 1};\phi]/(x_2^3-x_3^2)\]
	where $\phi:x_i\to \eta^ix_i$.

	As a coalgebra, $g$ is grouplike and $x_i$ are such that $\Delta(x_i)=x_i\otimes 1+g^i\otimes x_i$
	and $\varepsilon(x_i)=0$. Then we get $S(x_i)=-g^{-i}x_i.$
\end{ex}
\subsection{HW for the day}
\begin{prob}
	Let $A$ and $B$ be rings. Let $_AM$, $_AN_B$, and $_BC$. Then 
	\[\Hom_A(N\otimes_B C,M)\cong \Hom)B(C, \Hom_A(N,M))\]
	and if instead $C_A$, $_AN_B$ and $M_B$, then you can do things over $A^{op}$ and $B^{op}$.
\end{prob}

\section{February 25, 2019}
Topic for today: \textbf{Hopf actions on algebras.} One can naturally extend everything we talk about
today to (co)actions on (co)algebras. :)
\subsection{Example of the Day}
\begin{ex}\label{ex-22519}
	$V=k\langle g,g^{-1},x\rangle/(gg^{-1}=1=g^{-1}g)$. Then $g$ is grouplike and $\Delta(x)=x\otimes1+g\otimes x$, 
	$\varepsilon(x)=0$, and $S(x)=-g^{-1}x$.

	We can see that $\gldim V=1$ (use resolution $0\to V\oplus V\to V\to {_V k}\to 0$). Note that
	it is NOT AS-Gorenstein. $\Ext_V^1(k,V)$ is infinite dimensional over $k$ (not easy!).
\end{ex}
\begin{rmk}
	We've seen elements like $x$ above several times -- these elements are called $(1,g)$-primitive 
	(generalizing the concept of primitive, obviously). 

	Also, these are examples of \textbf{pointed Hopf algebras}(!!) These are $V$ such that (as a
	coalgebra) the coradical is a sum of one-dimensional modules. This is actually related to the fact 
	that every simple submodule of $V$ is of the form $kg^n$ (in this case $n=1$).
\end{rmk}
\subsection{HW of the Day}
\begin{prob}
	This problem considers (one of the) internal $\Hom$(s) in $_V\calM$. Suppose $V$ has bijective 
	antipode. Then for any $M$ and $N$,
	\[\Hom^{S^{-1}}(M,N):=\Hom_k(M,N)\]
	where a $V$ action is defined by (for $v\in V$ and $f\in\Hom^{S^{-1}}(M,N)$) 
	\[(v\cdot f)(m)=\sum v_2f(S^{-1}(v_1)m)\]
	where, as usual, $\Delta v=\sum v_1\otimes v_2$.
	\begin{enumerate}
		\item Show that this makes $\Hom^{S^{-1}}(M,N)$ into a left $V$-module.
		\item Show that $\Hom^{S^{-1}}(-,-)$ is right adjoint to $-\otimes_k -$.
	\end{enumerate}
\end{prob}
\subsection{Automorphisms and Group Actions}
Let $A$ be an algebra. $\Aut_{alg}(A)$ denotes the \textbf{group of autormorphisms of $A$.}
\begin{defn}\label{def-action}
	Let $G$ be a group. We say $G$ acts on $A$ ($G\curvearrowright A$) fi there is a group homomorphism
	$G\to \Aut_{alg}(A)$.
\end{defn}
\begin{defn}
	Alternatively, we say $G\curvearrowright A$ if
	\begin{itemize}
		\item $A$ is a $G$-module.
		\item $\forall g\in G$, $g(1_A)=1_A$
		\item $\forall a,b\in A$, $g(ab)=g(a)g(b)$
		\item $g$ is bijective.
	\end{itemize}
\end{defn}
\begin{defn}
	Suppse $G\curvearrowright A$. Then the \textbf{invariant subring} of this action is 
	\[A^G:=\{a\in A|g(a)=a,\forall g\in G\}\subseteq A\]
\end{defn}
This was fairly popular around 1880-1960 or so.
\begin{ex}
	$A=k[x_1,x_2]$, $G=\langle\sigma\rangle\cong C_2$ where
	$\sigma\cdot f(x_i)=f(-x_i)$. One can compute that this just switches signs on monomial 
	of odd degree (and fixes the even ones) so
	\[A^G=\left\{\sum c_{ij}x_1^ix_2^j|i+j=2k\right\}\cong k[a,b,c]/(ac-b^2)\]
\end{ex}
Generalizing this, many of the questions of this time revolved around asking what category of 
algebra $k[x_1,\dots,x_n]^G$ is for arbitrary $G$. We may return to this later.

\subsection{Derivations}
\begin{defn}
	Let $A$ be an algebra. A map $d\in\Hom_k(A,A)$ is called a \textbf{derivation} if
	it satisfies the Leibniz rule.
\end{defn}
\begin{lem}
	The collection of all derivations of an algebra from a Lie algebra -- denoted $\Der(A)$.
\end{lem}
\begin{defn}
	Let $L$ be a Lie algebra. We say $L$ acts on $A$ if there is a Lie algebra morphism $L\to \Der(A)$.
	Equivalently, if
	\begin{enumerate}
		\item $A$ is a left $L$-module
		\item $\forall l\in L$, $l(1_A)=0$
		\item $\forall a,b\in A$, $l(ab)=l(a)b+al(b)$.
	\end{enumerate}
\end{defn}
\begin{rmk}
	The second item above is implied by the first (in fact, James included a similar statement in 
	the definition of a derivation) but apparently there is a reason to include it!
\end{rmk}
\begin{ex}
	$A=k[x_1,x_2]$, $d=\frac{\partial}{\partial x_1}$. Then $L=kd$ is a Lie algebra that acts on $A$.

	Then $A^L$ (see below) is $k[x_2]$
\end{ex}
\begin{defn}
	If $L$ is a Lie algebra acting on $A$, then the invariant subring under the action is
	\[A^L=\{a\in A|l(a)=0,\forall l\in L\}\]
\end{defn}

\subsection{Putting These Things Together}
Let $A$ be an algebra and $V$ be a Hopf algebra.
\begin{defn}
	We say $V$ actions on $A$ if
	\begin{itemize}
		\item $A$ is a left $V$-module
		\item $A$ is an algebra object in $({_V\cal M},\otimes)$
	\end{itemize}
	Note that the second item above implies the first!

	Equivalently, $V$ acts on $A$ if
	\begin{itemize}
		\item $A\in {_V\cal M}$
		\item For all $v\in V$, $v\cdot(1_A)=\varepsilon(v)1_A$
		\item For all $a,b\in A$, $v(ab)=\sum v_1(a)v_2(b)$.
	\end{itemize}
\end{defn}

Returning to the example from the beginning of class (\ref{ex-22519}), if $V$ acts on $A$,
then 
\begin{itemize}
	\item $g\in \Aut_{alg}(A)$
	\item $x\in\Der_g(A)$
\end{itemize}
In particular, notice that $x(1)=0$ and $x(ab)=x(a)b+g(a)x(b)$.

\section{March 11, 2019}
We took a week off since James was in Germany for a conference. We're coming back together for this
last week of the quarter.

Here is one result that was brought up at the conference. It was proved by a grad student and might 
have some missing hypotheses:
\begin{thm}
	If $\ch k=0$ (and some other mild hypotheses) then the Etingof-Ostrik conjecture holds for $H$ 
	if and only if it holds for $H^*$. That is,
	\[\operatorname{kdim}\oplus \Ext_H^i(k,k)=\operatorname{kdim}\oplus \Ext^i_{H^*}(k,k)\]
	where $\operatorname{kdim}$ means Krull dimension.
\end{thm}

\subsection{Example of the Day}
\begin{ex}[Small Quantum Group of $\sl_2$]
	Let $q\in k$ be a primitive $l^{th}$ root of unity ($l\ge 2$). Then
	\[u_q(\sl_2)=k\langle K,K^{-1},E,F\rangle/R\]
	where $R$ is the ideal for the relations
	\begin{align*}
		KK^{-1}=K^{-1}K=1\\
		KEK^{-1}=q^2E\\
		KFK^{-1}=q^{-2}F\\
		EF-FE=\frac{K-K^{-1}}{q-q^{-1}}\\
		E^l=F^l=K^0\\
		K^l=1
	\end{align*}
	Now $\dim_k u_q(\sl2)=l^3$ and this is (symmetric) Frobenius and noncommutative and noncocommutative.
	As a coalgebra, $K$ is grouplike and $E,F$ are almost primitive:
	\[\Delta E=E\otimes 1+K\otimes E,\quad \Delta F = F\times K^{-1}+1\otimes F\]
	and $S(E)=-K^{-1}E$ and $S(F)=-FK$.

	Then $\sl_2=ke\oplus kf\oplus kh$ and there is a map $e\mapsto E$, $f\mapsto F$ and $K=\exp(h)$. Hmm
\end{ex}

\subsection{Last Lecture}
I wasn't here, so I missed this part. Let $A$ be a Noetherian connected graded AS-regular algebra and let
$V$ be a Hopf algebra action on $A$ such that each $A_i$ is a left $V$-module.
\begin{enumerate}
	\item $k:=A/A_{\ge 1}$ and $\oplus_0^\infty \Ext_A^i(k,k)$ is an algebra.
	\item (S. Paul Smith's Result) $\oplus \Ext_A^i(k,k)$ is Frobenius and $\Ext_A^d(k,k)$ is one
	dimensional over $k$ when $d=\gldim A$.
	\item  Each $\Ext^i_A(k,k)$ is a left $V$ module.
	\item $(\oplus \Ext_A^i(k,k))^{op}$ is a left $V$ algebra.
\end{enumerate}

Now pick any nonzero element $e\in \Ext_A^d(k,k)$ where $d$ is as above. For all $v\in V$, 
$v\cdot e\in \Ext_A^d(k,k)$ and we write $v\cdot e=\operatorname{hdet}(v)v$ for $\operatorname{hdet}(v)\in k$.
\begin{defn}
	A map $\operatorname{hdet}:V\to k$ that fits in as above is called the \textbf{homological determinant}
	of the $V$-action on $A$.
\end{defn}
\begin{lem}
	$\operatorname{hdet}$ is an algebra map.
\end{lem}
\begin{lem}
	If $A=k[x_1,\dots,x_n]$, $V=kG$, $G\subseteq GL(n)$ then
	\[\operatorname{hdet} g = \det g\]
	for all $g\in G$.
\end{lem}
\begin{defn}
	We say that $\operatorname{hdet}$ is trivial if $\operatorname{hdet}(v)=\varepsilon(v)$ for all $v\in V$.
\end{defn}

Recall the theorem
\begin{thm}[Watanabe]
	Suppose $G$ is a finite subgroup in $GL_n$. If $\deg g=1$ for all $g\in G$, then $k[x_1,\dots,x_n]^G$ is Gorenstein.
\end{thm}
\begin{thm}[Noncommutative Watanabe]
	Let $A$ be a Noetherian connected graded AS-regular algebra. Let $V$ be a semisimple Hopf algebra acting on 
	$A$ such that each $A)i$ is a left $V$-module. If $\operatorname{hdet}$ is trivial, then $A^V$ is AS Gorenstein.
\end{thm}
\begin{rmk}
	This last result is important because it distinguishes between the commutative and noncommutative cases: in the commutative
	case the only AS-regular ring is the polynomial ring. This shows us that there is in fact a vast class of them 
	in the noncommutative case. In fact it's even bigger!
\end{rmk}
Recall that $g\in GL_n$ is a reflection if $g$ is similar to $\operatorname{diag}(1,\dots,1,\sqrt{-1})$. Let $R$ be the subgroup of $G$ generated
by the reflections in $G$. 
\begin{thm}[Watanabe II]
	Suppose $G$ is a finite subgroup of $GL_n$. Then $k[x_1,\dots,x_n]^G$ is Gorenstein if and only if
	$\det \bar g=1$ for all $g\in G/R$.
\end{thm}

\subsection{Noncommutative Reflections}
\begin{defn}
	Let $g\in\Aut_{gralg}(A)$. Then
	\[\operatorname{Tr}(g)=\sum_0^\infty \operatorname{tr}(g|_{A_i})t^i\in k[[t]].\]
\end{defn}
Then a result (which?) gets us that $\operatorname{Tr}(g)=\frac{1}{p(t)}$ for a polynomial $p$.
Consider $\tilde A=k_{p_{ij}}[x_1,\dots,x_n]$ where $x_jx_i=p_{ij}x_ix_j$ for all $i<j$. Then 
\begin{defn}
	$g\in\Aut_{gralg}(\tilde A)$ is a reflection if $\operatorname{Tr}(g)=\frac{1}{(1-t)^{n-1}(1-\lambda t)}$.
\end{defn}

\section{March 13, 2019}
Today we are going to discuss why considering algebras over $A$ is not quite sufficient in many cases --
why we may be instead interested in studying $_A\calM$.

Let $A$ be an algebra and $V$ a Hopf algebra. Let $V$ act on $A$ and define
\[A^V=\{a\in A|va=\varepsilon(v)a,\forall v\in V\}\]
We can see that we get $A^v=\Hom_V({_Vk},A)$ and then use this to compute the cohomology
\[\Ext_V^i(k,A)\]

We forgot the
\subsection{(Non)-Examples of the Day}
\begin{ex}
	Let $A$ be an algebra, $n\ge 2$ an integer. Then $M_n(A)$ can't be a Hopf algebra.

	The idea here is that there is no counit $\varepsilon:M_n(A)\to k$.
\end{ex}
\begin{ex}
	If $V_1$ and $V_2$ are Hopf algebras, then $V_1\oplus V_2$ is not a Hopf algebra.
	Notice here that it is both an algebra and a coalgebra!

	Here, use that $1_V=1_{V_1}+1_{V_2}$ and compute $\Delta(1_V)\ne 1_V\otimes 1_V$.
\end{ex}
\begin{ex}
	If $A$ is finite dimensional and not Frobenius, then $A$ can't be a Hopf algebra.
\end{ex}

\subsection{Some Category Theory}
Let $k$ be af field.
\begin{defn}
	Let $\mathscr C$ be a category. $\mathscr C$ is called \textbf{$k$-linear} if
	\begin{itemize}
		\item $\Hom_{\mathscr C}(M,N)\in\Vectk$ for all $M$ and $N$ in $\mathscr C$
		\item Composition is $k$-linear.
	\end{itemize}
\end{defn}
\begin{defn}
	A category $\mathscr C$ is called \textbf{finite} if $\mathscr C\cong {_A\calM_{fd}}$ 
	for some finite dimensional algebra $A$.
\end{defn}
\begin{defn}
	A category $\mathscr C$ is called \textbf{monoidal} (notice that sometimes this books 
	call this a tensor category) if it is equipped with
	\begin{itemize}
		\item A bifunctor $-\otimes -:\mathscr C\times\mathscr C\to \mathscr C$
		\item An associativity isomorphism: for all $X,Y,X\in\mathscr C$:
		\[a_{X,Y,Z}:(X\otimes Y)\otimes Z\xrightarrow{\sim} X\otimes(Y\otimes Z)\]
		\item An identity object $\mathbf{1}\in\mathscr{C}$
		\item An isomorphism $i:\mathbf{1}\otimes\mathbf{1}\to \mathbf{1}.$
	\end{itemize}
	That satisfies the pentagon axiom (Figure~\ref{fig:pentagon}) and the unit axioms
	\[L_\mathbf{1}X\xrightarrow{\sim} \mathbf{1}\otimes X,\qquad R_\mathbf{1}:X\xrightarrow{\sim}X\otimes \mathbf{1}\]
\end{defn}
\begin{figure}\label{fig:pentagon}
	\centering
	\begin{tikzcd}
		((W\otimes X)\otimes Y)\otimes Z\ar[d,"a_{W,X,Y}\otimes \id_Z"]\ar[r,"a_{W\otimes X,Y,Z}"] & (W\otimes X)\otimes (Y\otimes Z)\ar[dd,"a_{W,X,Y\otimes Z}"]\\
		(W\otimes(X\otimes Y))\otimes Z \ar[d,"a_{W,X\otimes Y,Z}"]& \\
		W\otimes((X\otimes Y)\otimes Z)\ar[r,"\id_W\otimes a_{X,Y,Z}"] & W\otimes(X\otimes (Y\otimes Z))
	\end{tikzcd}
	\caption{The Pentagon Axioms}
\end{figure}
\begin{ex}
	Let $A$ be an algebra and let $\mathscr C= ({_A\calM_A},\otimes_A,\mathbf{1}=A)$.
	Clearly this works. :)
\end{ex}

Recall that if $V$ is a Hopf algebra and $M\in {_V\calM_{fd}}$, then $M^*:=\Hom_k(M,k)\in\calM_V$
then we defined $M^{*S}=\tilde S\circ M^*\in {_V\calM}$. This is the right dual.

We can also define the left dual if $S$ is bijective. In this case, we define $M^{*S^{-1}}=\tilde S^{-1}\circ M^*\in{_V\calM}$.

Now let $(\mathscr C,\otimes, \mathbf{1})$ be a monoidal category.
\begin{defn}
	Let $X\in\mathscr C$. An object $X^*$ in $\mathscr C$ is said to be a \textbf{left dual of $X$}
	if there exist an \textbf{evaluation map}
	\[\operatorname{ev}_X:X^*\otimes X\to \mathbf{1}\]
	and a \textbf{coevaluation map}
	\[\operatorname{coev}_X:\mathbf{1}\to X\otimes X^*\]
	such that
	\[X\xrightarrow{\operatorname{coev}_X\otimes\id_X}(X\otimes X^*)\otimes X\to X\otimes (X^*\otimes X)\xrightarrow{\id_X\otimes \operatorname{ev}_X} X\]
	is an isomorphism as well as the map ``on the other side''.
\end{defn}
\begin{defn}
	In the context of the above, a \textbf{right dual of $X\in\mathscr C$} is a left dual of $X$ in $\mathscr C^{op}$.
	The right dual is denoted (confusingly) by ${^*X}$.
\end{defn}
So in the Hopf algebra case, ${^*M}=M^{*S}$ and $M^*=M^{*S^{-1}}.$
\begin{defn}
	A category $\mathscr C$ is called \textbf{rigid} if every $X\in\mathscr C$ has both left and right duals.
\end{defn}
\begin{defn}
	$\mathscr C$ is called a \textbf{tensor category} if 
	\begin{itemize}
		\item $\mathscr C$ is finite (this implies $k$-linear)
		\item It is monoidal
		\item it is rigid
		\item $\End_\mathscr{C}(\mathbf{1})=k$
	\end{itemize}
\end{defn}
\begin{rmk}
	If we remove the last condition and replace it with $\End(\mathbf{1})=\oplus k$, then
	we call $\mathscr C$ \textbf{multi-tensor.}
\end{rmk}
\begin{defn}
	$\mathscr C$ is called \textbf{fusion} if 
	\begin{itemize}
		\item It is tensor
		\item It is semisimple (every object is a direct sum of simple objects).
	\end{itemize}
\end{defn}
\begin{rmk}
	If instead $\mathscr C$ is multitensor and semisimple, then it is called \textbf{multifusion.}
\end{rmk}
\begin{ex}
	\begin{enumerate}
		\item $\Vectk$ is an example.
		\item If $V$ is a finite dimensional Hopf algebra, ${_V\calM}$ is tensor.
		\item If $V$ is also semisimple, ${_V\calM}$ is fusion.
	\end{enumerate}
\end{ex}
Recall this conjecture for the first day (Etinghof-Ostrik restated):
\begin{conj}
	If $\mathscr C$ is a tensor category then $\Ext^*(\mathbf{1},\mathbf{1})$ is Noetherian.
\end{conj}

\section{March 15, 2019}
On Wednesday, we said that if $M\in{_V\calM_{fd}}$ that $M^*=M^{*S^{-1}}$ and ${^*M}=M^{*S}$,
but in fact these are reversed.
\subsection{Example of the Day}
\begin{ex}
	Usually we have been defining Hopf algebras over vector spaces, but this will be an object in $\text{\sc grVect}_k$,
	that is $V=\oplus_{i\in\Z}V_i$ and if $M=\oplus M_i$ and $N=\oplus N_i$, then $M\otimes N=\oplus_i(\oplus_{s+t=i}M_s\otimes N_t)$.

	Now putting the ``symmetric'' back into the monoidal category, we define $\tau:M\otimes N\to N\otimes M$ via
	\[m\otimes n\mapsto (-1)^{|m||n|}n\otimes m.\]
	Then we have our symmetric monoidal category and we can define a Hopf algebra in it.

	$V=\Lambda(x_1,\dots,x_n)$, the exterior algebra on $n$ elements. Define the coalgebra structure 
	so that each $x_i$ is primitive. Note that here $V$ is a Hopf algebra over graded vector spaces, but 
	NOT over vector spaces (unless $\ch k=2$).

	To see this, we can compute 
	\[0=\Delta(x_i^2)=\Delta(x_i)^2=(x_i\otimes 1+1\otimes x)^2=x_i^2\otimes 1+x_i\otimes x_i+\tau(x_i\otimes x_i)+1\otimes x_i^2\]
	and if $\tau$ is the usual vector space twist (and the characteristic isn't 2) we get a contradiction.
\end{ex}

\subsection{Some History}
Before 1941 people studied rings like $kG, \mathcal U(\g)$ and $\mathcal O(G)$. An open question at that
point was ``If $\mathcal U(\g)$ is Noetherian, then is $\dim \g<\infty?$''

In 1941, Hopf algebras were born, although they were originally defined (by Hopf) to be over graded
vector spaces. Eventually people started considering them over vector spaces.

In 1967, Sweedler wrote the first book about Hopf algebra over vector spaces. Then later, near 1970,
Rota began working more on combinatorial Hopf algebras. It ends up in this context that often the 
M\"obius function ends up being the co(unit? multiplication?).

In 1975, Kaplansky posited 10 conjectures. Of those only one more remains:
\begin{conj}{Kaplansky, `75 (\#6)}
	If $V$ is a semisimple Hopf algebra and $M$ is a simple $V$-module, then 
	\[\dim_kM|\dim_kV.\]
\end{conj}

From 1960-90, the studies of $kG, \mathcal U(G)$ and $\mathcal O(G)$ and semisimple Hopf algebra continued,
led in large part by Radford. In 1987, Drinfeld and Jimbo came out with quantum groups.
At the same time Drinfeld introduced quasi- and braided Hopf algebras and weak Hopf algebras.

Of course we all know about Nichols algebras. Also we switched to different tensor categories, including
fusion, pivotal, and spherical ones.

In 1993 Lusztig invented the small quantum groups, which are of particular interest to representation theorists.
`97 was Brown-Goodearl. Then around 1990 there were lots of classification projects including
the one by Anruskiewitsch-Schneider project. Near this time Zhu proved that every Hopf algebra of dimension $p$
is a group algebra $k\Z/p$. 

In 2004 was Etingof-Ostrik. The next year, Etingof-Nikshych-Ostrik wrote a big paper ``On Fusion Categories''
which really shifted the perspective from Hopf algebras to things in fusion categories, where many of the big
results were transferred.

In 2015 Etingof, Nikshych, Ostrik (and someone else) wrote ``Tensor Categories''.

\subsection{Extensions}
We can look at extensins of Brown-Goodearl. The question ``Is every Noetherian X Gorenstein?'' can be extended to be asked about any noun 
in \{Hopf Algebras, weak Hopf algebras, braided hopf algebras, Nichols algebgras, quasi-Hopf algebras\}.

We can also consider extensions of Etingof-Ostrik: ``Let $V$ be a finite dimensional X and $A$ be a 
Notherian left $V$-module such that $A^V$ is Noetherian. Is then $\Ext_V^*(k,A)$ Noetherian?'' Again
we can take $X$ to be anything in the above set.

In 2019, who knows!

\end{document}
