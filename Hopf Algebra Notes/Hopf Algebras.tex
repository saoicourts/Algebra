\documentclass[12pt]{article}

\usepackage{setspace}

\usepackage{amsmath, amsfonts, amssymb, graphicx, color, fancyhdr, lipsum, scalerel, stackengine, mathrsfs, tikz-cd, mdframed, enumitem, framed, adjustbox, bm, upgreek, x	color}
\usepackage[framed,thmmarks]{ntheorem}
\usepackage[mathscr]{euscript}

%set up theorem/definition/etc envs
%Problems will be created using their own counter and style
\theoremstyle{break}
\theoreminframepreskip{0pt}
\theoreminframepostskip{0pt}
\newframedtheorem{prob}{Problem}[section]

%solution template
\theoremstyle{nonumberbreak}
\theoremindent0.5cm
\theorembodyfont{\upshape}
\theoremseparator{:}
\theoremsymbol{\ensuremath\spadesuit}
\newtheorem{sol}{Solution}

%Theorems
\definecolor{thmcol}{RGB}{0,102,34}
\theoremstyle{changebreak}
\theoremseparator{}
\theoremsymbol{}
\theoremindent0.5cm
\theoremheaderfont{\color{thmcol}\bfseries} 
\newtheorem{thm}{Theorem}[subsection]

%Lemmas and Corollaries
\theoremheaderfont{\bfseries}
\newtheorem{lem}[thm]{Lemma}
\newtheorem{cor}[thm]{Corollary}

%Create a new env that references a theorem and creates a 'primed' version
%Note this can be used recursively to get double, triple, etc primes
\newenvironment{thm-prime}[1]
  {\renewcommand{\thethm}{\ref{#1}$'$}%
   \addtocounter{thm}{-1}%
   \begin{thm}}
  {\end{thm}}

\setlength\fboxsep{15pt}

%Shade definitions
\theoremindent0cm
\theoremheaderfont{\normalfont\bfseries} 
\def\theoremframecommand{\colorbox[rgb]{.85,1,.85}}
\newshadedtheorem{defn}[thm]{Definition}

%Example
\theoremstyle{break}
\def\theoremframecommand{\colorbox[rgb]{0.9,0.9,0.9}}
\newshadedtheorem{ex}{Example}[section]

%Man, that's really good! Let's use the same thing for definitons.
\newenvironment{def-prime}[1]
  {\renewcommand{\thethm}{\ref{#1}$'$}%
   \addtocounter{thm}{-1}%
   \begin{def}}
  {\end{def}}

%proofs
\theoremstyle{nonumberbreak}
\theoremindent0.5cm
\theoremheaderfont{\sc}
\theoremseparator{}
\theoremsymbol{\ensuremath\spadesuit}
\newtheorem{prf}{Proof}

\theoremstyle{nonumberplain}
\theoremseparator{:}
\theoremsymbol{}
\newtheorem{conj}{Conjecture}

%remarks
\theoremstyle{change}
\theoremindent0.5cm
\theoremheaderfont{\sc}
\theoremseparator{:}
\theoremsymbol{}
\newtheorem{rmk}[thm]{Remark}

%Replacement for the old geometry package
\usepackage{fullpage}

%Put page breaks before each part
\let\oldpart\part%
\renewcommand{\part}{\clearpage\oldpart}%

%Center each figure by default
\makeatletter
\g@addto@macro\@floatboxreset{\centering}
\makeatother

%header stuff
\setlength{\headsep}{24pt}  % space between header and text
\pagestyle{fancy}     % set pagestyle for document
\lhead{Notes on Hopf Algebras} % put text in header (left side)
\rhead{Nico Courts} % put text in header (right side)
\cfoot{\itshape p. \thepage}
\setlength{\headheight}{15pt}
\allowdisplaybreaks

%Set of Integers
\newcommand*{\Z}{
\mathbb{Z}
}
%Set of Natural Numbers
\newcommand*{\N}{
\mathbb{N}
}
%Set of Real Numbers
\newcommand*{\R}{
\mathbb{R}
}
%Set of Complex Numbers
\newcommand*{\C}{
\mathbb{C}
}
%Rationals
\newcommand*{\Q}{
\mathbb{Q}
}

%Section break
\newcommand*{\brk}{
\rule{2in}{.1pt}
}

\DeclareMathOperator{\Aut}{Aut}

%raise that Chi!
\DeclareRobustCommand{\Chi}{{\mathpalette\irchi\relax}}
\newcommand{\irchi}[2]{\raisebox{\depth}{$#1\chi$}} 

%Image
\DeclareMathOperator{\im}{Im}

\DeclareMathOperator{\Ext}{Ext}

%Coker
\DeclareMathOperator{\coker}{coker}

%characteristic
\DeclareMathOperator{\ch}{char}

%rank
\DeclareMathOperator{\rank}{rank}

%identity map
\DeclareMathOperator{\id}{id}

%Hopf algebra stuff
\newcommand*{\Vectk}{\operatorname{Vect}_k}
\newcommand*{\Algk}{\operatorname{Alg}_k}
\newcommand*{\Coalgk}{\operatorname{Coalg}_k}

%fix tilde
\let\tilde\relax
\newcommand*{\tilde}[1]{\widetilde{#1}}

% Enumerate will automatically use letters (e.g. part a,b,c,...)
\setenumerate[0]{label=(\alph*)}

\begin{document}
%make the title page
\title{Hopf Algebras\vspace{-1ex}}
\author{A course by: Prof. James Zhang\\
Notes by: Nico Courts}
\date{Winter 2019}
\maketitle

\renewcommand{\abstractname}{Introduction}
\begin{abstract}
	These are the notes I took in class during the Winter 2019 topics course
	\textit{Math 582H - Hopf Algebras} at University of Washington, Seattle. 
	
	The course description follows:

	\brk

	This course is an introduction to Hopf algebras. In addition to basic material in 
	Hopf algebra, we will present some latest developments in quantum groups and tensor 
	and fusion categories. One of the newer topics is homological properties of 
	Noetherian Hopf algebras of low Gelfand-Kirillov dimension. A good reference for 
	the first two topics in the book \textit{Hopf Algebras and Their Action Rings} by 
	Susan Montgomery. Here is a list of possible topics:

\begin{itemize}
	\item Classical theorems concerning finite dimensional Hopf algebras.
	\item Infinite dimensional Hopf algebras and quantum groups.
	\item Duality and Calabi-Yau property.
	\item Actions of Hopf algebras and invariant theory.
	\item Representations of Hopf algebras, tensor and fusion categories.
\end{itemize}
\end{abstract}

\section{January 7, 2019}
If you don't know what a symmetric tensor category is, today is going to be a three 
star day. Max is 5.

\subsection{Overview}
We are shooting to understand two conjectures:

\begin{conj}[Etingof-Ostrik `04]
	If $A$ is a finite dimentional Hopf algebra, then 
	\[\bigoplus_{i\ge 0}\Ext_A^i(_Ak, _Ak)\]
	is Noetherian.
\end{conj}

\begin{conj}[Brown-Goodearl `98]
	If $A$ is a Noetherian Hopf algebra, then the injective dimension of $A_A$ is finite.
\end{conj}

These are both still open! In fact there is a meeting at Oberwolfach this March concerning
exactly these conjectures.

\subsection{Symmetric Tensor Categories}
We are going to be using the following notation throughout:
\begin{itemize}
	\item $k$ is a field
	\item $\Vectk$ is the category of $k$-vector spaces
	\begin{itemize}
		\item $\Vectk$ is closed under tensor products
		\item There is an element $k\in\Vectk$ such that
		\[k\otimes_k V\cong V\cong V\otimes_k k\]
		where the above isomorphisms are natural.
		\item $V\otimes_k W\cong W\otimes_k V$
	\end{itemize}
	\item An algebra is an object in $\Vectk$.
\end{itemize}

\begin{defn}\label{def-alg}
	$V\in\Vectk$ is called an \textbf{algebra object} if there are two morphisms
	\begin{enumerate}
		\item $m:V\otimes V\to V$
		\item $u:k\to V$
	\end{enumerate}
	such that the diagrams in figure \ref{fig-alg} commute.
	
\end{defn}
\begin{figure}\label{fig-alg}
	\begin{tikzcd}
		V\otimes V\otimes V\ar[r,"\id_V\otimes m"]\ar[d,"m\otimes \id_V"] & V\otimes V\ar[d,"m"]\\
		V\otimes V \ar[r,"m"] & V
	\end{tikzcd}
	\begin{tikzcd}
		k\otimes V\ar[r,"\sim"]\ar[rd,swap,"u\otimes\id_V"] & V & V\otimes k\ar[l,"\sim",swap]\ar[dl, "\id_V\otimes u"]\\
		& V\otimes V \ar[u,"m"] &
	\end{tikzcd}
	\caption{Diagrams for definition \ref{def-alg}.}
\end{figure}

\begin{lem}
	$V\in\Vectk$ is an algebra object iff $V$ is an algebra over $k$.
\end{lem}

\begin{lem}
	If $C$ is a symmetric tensor category, so is $C^{op}$.
\end{lem}

Then the natural thing to ask is: what is an algebra object in this opposite category?
\begin{defn}
	A \textbf{coalgebra object} in $C$ is an algebra object in $C^{op}$. Here we have 
	comultiplication $\Delta$ and counit $\varepsilon$.
\end{defn}
\begin{rmk}
	Naturally you could go about defining this from first principles and drawing the diagrams
	in figure \ref{fig-alg} with the arrows reversed, but we are probably mature enough
	to do without that (saving my fingers from repetitive strain injury in the process.)
\end{rmk}

\begin{lem}
	$\Algk$, defined as the category of algebra objects in $\Vectk$, is a symmetric tensor category. Furthermore
	$\Coalgk$, the category of coalgebra objects in $\Vectk$, is a symmetric tensor category.
\end{lem}

\begin{lem}
	The following are equivalent:
	\begin{enumerate}
		\item $V$ is an algebra object in $\Coalgk$
		\item $V$ is a coalgebra object in $\Algk$
		\item There are morphisms $m,u,\Delta,\varepsilon$ such that
		\begin{itemize}
			\item $(V,m,u)$ is an algebra
			\item $(V,\Delta,\varepsilon)$ is a coalgebra
			\item Equivalently:
			\begin{itemize}
				\item $m$ and $u$ are coalgebra morphisms
				\item $\Delta$ and $\varepsilon$ are algebra morphisms.
			\end{itemize}
		\end{itemize}
	\end{enumerate}
\end{lem}
\begin{prf}
	The nice thing here is that the $(a)\Leftrightarrow (c)$ without the last condition. A similar fact holds
	for $(b)$ except the second-to-last. The last thing to do is to prove the last two conditions are equivalent.
\end{prf}
\begin{prob}
	Fill in the details for the proof above.
\end{prob}
\begin{sol}
	Assume that $(V,m,u,\Delta,\varepsilon)$ is an algebra and coalgebra and further that
	$m$ and $u$ are coalgebra morphisms. That means in particular that the diagrams in figure \ref{fig-coalg-mor} commute.
	\begin{figure}[h]\label{fig-coalg-mor}
		\centering
		\begin{tikzcd}
			V\otimes V \ar[d,swap,"\Delta\otimes\Delta"]\ar[r,"m"] & V\ar[r,"\Delta"] & V\otimes V\\
			V^{\otimes 4}\ar[rr,"T_{2,3}"] & & V^{\otimes 4}\ar[u,swap,"m\otimes m"]
		\end{tikzcd}
		\begin{tikzcd}
			k\ar[r,"\Delta"]\ar[d,"u"] & k\otimes k\ar[d,"u\otimes u"]\\
			V\ar[r,"\Delta"] & V\otimes V
		\end{tikzcd}
		\label{fig-coalg-mor}
		\caption{$m$ and $u$ are coalgebra morphisms.}
	\end{figure}

	We are looking to prove that $\Delta$ and $\varepsilon$ are algebra morphisms, or that 
	the diagrams in figure \ref{fig-coalg-mor}  commute.
	\begin{figure}[h]
		\centering
		\begin{tikzcd}
			V\otimes V\ar[r,"m"]\ar[d,"\Delta\otimes \Delta"] & V\ar[d,"\Delta"]\\
			V^{\otimes 4}\ar[r,"m"] & V\otimes V
		\end{tikzcd}
		\begin{tikzcd}
			V\otimes V\ar[r,"\varepsilon\otimes\varepsilon"]\ar[d,"m"] & k\otimes k\ar[d,"m"]\\
			V\ar[r,"\varepsilon"] & k
		\end{tikzcd}
		\label{fig-alg-mor}
		\caption{$\Delta$ and $\varepsilon$ are algebra morphisms.}
	\end{figure}

	From here it's actually a bit boring because it's kinda just a definition/notation game.
	It boils down to the fact that the (co)multiplication on $V\otimes V$ has a twist that 
	exactly lines up so that each square is saying the same thing.
\end{sol}

\begin{defn}
	$V$ is called a \textbf{bialgebra object} if $V$ is an algebra object in $\Coalgk$.
\end{defn}

\begin{prob}
	\begin{enumerate}
		\item Suppose that $\ch k\ne 2$. Classify all bialgebras of $\dim 2.$
		\item Do the same for $\ch k = 2.$
	\end{enumerate}
\end{prob}
\begin{sol}
	\subsubsection*{Part (a)}
	Consider $\varepsilon:V\to k$ and consider $\ker\varepsilon\lhd V$. By rank-nullity, $\dim\ker\varepsilon= 1$, so
	$\ker\varepsilon=kx$ for some $x\in V$. Therefore $x^2=cx$ for some $c$, and if $c=0$, then (as an algebra) $V\cong k[x]/(x^2)$.
	Otherwise consider $y=\frac{x}{c}$. In this case $y^2=\frac{x^2}{c^2}=\frac{x}{c}=y$, and $V\cong k[x]/(x^2-x)$.

	Notice that in either case $\varepsilon(x)=0$, so let
	\[\Delta(x)=a(1\otimes 1)+b(1\otimes x)+c(x\otimes 1) + d(x\otimes x)\]
	and using that $\varepsilon\otimes \id\circ\Delta=\id\otimes\varepsilon\circ\Delta$
	and that each should be (essentially) the identity (this is just the diagram we saw before),
	we get $a=0$ and $b=c=1$. Thus the coalgebra structure of any Hopf algebra is given by 
	\[\varepsilon(x)=0,\quad \Delta(x)=1\otimes x+x\otimes 1+d(x\otimes x).\]

	Consider first the case when $x^2=0$. Then since comultiplication will be an algebra morphism,
	\[0=\Delta(x^2)=\Delta(x)^2=1\otimes x^2+x^2\otimes 1+d^2(x^2\otimes x^2)+2(x\otimes x)+2d(x\otimes x^2)+2d(x^2\otimes x)\]
	and since $x^2=0$, 
	\[0=2(x\otimes x).\]
	But $x\otimes x$ is a basis element of $V\otimes V$, so $V$ can only have this algebra structure
	when $\ch k=2$. We will return to this in the next part.

	So then $x^2=x$ and using the computation above,
	\[1\otimes x+x\otimes 1+d(x\otimes x)=\Delta(x)=\Delta(x^2)=1\otimes x+x\otimes 1+(d^2+4d+2)(x\otimes x)\]
	so
	\[(d^2+3d+2)(x\otimes x)=0\quad\Rightarrow\quad d^2+3d+2=(d+2)(d+1)=0\]
	and so either $d=-1$ or $d=-2$.

	One can verify that $\Delta$ is coassociative, so we can conclude that when $\ch k\ne 2$, 
	there are precisely two Hopf algebra structures with algebra structure $k[x]/(x^2-x)$ and
	comultiplication either
	\[\Delta(x)=1\otimes x+x\otimes 1-x\otimes x\quad\text{or}\quad \Delta(x)=1\otimes x+x\otimes 1-2(x\otimes x)\]

	\subsection*{(b)}
	Now assume that $\ch k=2$ and that $V\cong k[x]/(x^2-x)$ as an algebra. Then using the analysis above, 
	we see that we can choose comultiplication either
	\[\Delta(x)=x\otimes 1+1\otimes x\quad\text{or}\quad \Delta(x)=1\otimes x+x\otimes 1+x\otimes x.\]

	If instead $V\cong k[x]/(x^2)$, then \textit{any} value of $d$ will suffice, so there are a full $k$'s worth
	of Hopf algebra structures that can appear.
\end{sol} 

\section{January 9, 2019}
Today we are going to rely heavily on Sweedler notation. :) Notice that if we are looking at actual objects in the 
diagram for coassociativity, we get
\begin{figure}[h]
	\centering
	\begin{tikzcd}
		v\ar[r]\ar[d] & \sum v_{(1)}\otimes v_{(2)}\ar[d]\\
		\sum v_{(1)}\otimes v_{(2)}\ar[r] & \sum (v_{(1)})_{(1)}\otimes (v_{(1)})_{(2)}\otimes v_{(2)}=\sum v_{(1)}\otimes (v_{(2)})_{(1)}\otimes (v_{(2)})_{(2)}
	\end{tikzcd}
	\caption{Coassociativity on elements in Sweedler notation}
\end{figure}

\begin{ex}
	Let $G$ be a group and $kG$ be the group algebra. The algebra structure arises as normal
	where $g\cdot h$ comes from the structure on $G$. Then $\Delta(g)=g\otimes g$ and this extends linearly.

	But then if you consider $\Delta(\sum c_gg)$, notice that by the nature of tensors this is not unique!
	So we will just write
	\[\Delta\left(\sum c_gg\right)=\sum_Gc_g(g\otimes g)=\sum v_{(1)}\otimes v_{(2)}\]
\end{ex}

\subsection{Algebra structure on $V\otimes V$}
We said earlier on that $\Algk$ is a symmetric \textit{tensor} category. But how do we
define the multiplication on the tensor product?

Well it all comes from the twist! We define
\[m_{V\otimes W}=(m_V\otimes m_W)\circ(\id_V\otimes \tau_{2,3}\otimes \id_W)\]
where $\tau_{2,3}$ is the twist morphism.

More simply, $u_{V\otimes W}:k=k\otimes k\to V\otimes V$ simply defined by $u_V\otimes u_W$.

So then when we say that $\Delta$ is an algebra morphism, we are saying that for all $v,w\in V$
\[\Delta(vw)=\sum(vw)_{(1)}\otimes(vw)_{(2)}=(\sum v_{(1)}\otimes v_{(2)})(\sum w_{(1)}\otimes w_{(2)})=\sum v_{(1)}w_{(1)}\otimes v_{(2)}w_{(2)}\]

\subsection{Hopf Algebras}
Already to the good stuff!
\begin{defn}
	$V\in\Vectk$ is a \textbf{Hopf algebra} if $V$ is a bialgebra together with an \textbf{antipode}
	$S:V\to V$ satisfying
	\[(S,\id_V)\circ \Delta=\varepsilon=(\id_V,S)\circ\Delta\]
\end{defn}
\begin{conj}
	If $V\in\Vectk$ is a Noetherian Hopf algebra, then $S$ is bijective.
\end{conj}

\subsection{History and Motivation}
Hopf himself was a topologist, so this is the first context in which it arose. In the 1940's,
he began studying Hopf algebras over $\Z_2$ graded $k$ vector spaces. For instance, the 
cohomology ring of topological space $X$ with coefficients in $k$.

Later, in combinatorics, they ended popping up. Looking at rings of symmetric functions and other places
gave some interesting examples.

Then in group theory you can define a functor from groups to Hopf algebras by $F(G)=kG$
with the diagonal map. The antipode is just the inverse.

Then with Lie algebras, you can look at $\mathcal{U}(L)$, the universal enveloping algebra is a Hopf algebra.

Finally with algebraic groups (yay!) we take an algebraic group $G$ and consider the ring of functions on it, 
which is again a Hopf algebra.

\brk

Some ``cousins'' of Hopf algebras: quasi, weak, multiplier, ribbon, quasi-triangular, etc Hopf algebras. 
Each has slightly different base category or restrictions.

\end{document}
