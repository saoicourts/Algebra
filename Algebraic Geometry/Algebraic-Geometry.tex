\documentclass[12pt]{article}

\usepackage{setspace}

\usepackage{amsmath, graphicx, color, fancyhdr, tikz-cd, mdframed, enumitem, framed, adjustbox, bbm, upgreek, xcolor, hyperref, manfnt}
\usepackage[framed,thmmarks]{ntheorem}
\usepackage[style=alphabetic, bibencoding=utf8]{biblatex}
%Set the bibliography file
\bibliography{sources}

\usepackage[T1]{fontenc}
\usepackage[urw-garamond]{mathdesign}
\usepackage{garamondx}

%Replacement for the old geometry package
\usepackage{fullpage}

%Input my definitions
/home/nico/latex-includes/mydefs.tex

%Shade definitions
\theoremindent0cm
\theoremheaderfont{\normalfont\bfseries} 
\def\theoremframecommand{\colorbox[rgb]{0.9,1,.8}}
\newshadedtheorem{defn}[thm]{Definition}

%%%%%%%%%%%%%%%%%%%%%%%%%%%%%%%%%%%%%%%%%%%%%%%%%%%%%%%%%%%%%%%%%%%%%%
%%%%%%%%%%%%%%%%%%%%%%% Customize Below %%%%%%%%%%%%%%%%%%%%%%%%%%%%%%
%%%%%%%%%%%%%%%%%%%%%%%%%%%%%%%%%%%%%%%%%%%%%%%%%%%%%%%%%%%%%%%%%%%%%%

%header stuff
\setlength{\headsep}{24pt}  % space between header and text
\pagestyle{fancy}     % set pagestyle for document
\lhead{Algebraic Geometry} % put text in header (left side)
\rhead{Notes by Nico Courts} % put text in header (right side)
\cfoot{\itshape p. \thepage}
\setlength{\headheight}{15pt}
%\allowdisplaybreaks

% Document-Specific Macros
\DeclareMathOperator{\1}{\mathbbm{1}}
\DeclareMathOperator{\GL}{GL}
\DeclareMathOperator{\Sym}{Sym}

\begin{document}
%make the title page
\title{Algebraic Geometry\vspace{-1ex}}
\author{A course by Max Lieblich\\
Notes by Nico Courts}
\date{Autumn 2019/ Winter and Spring 2020}
\maketitle

\begin{abstract}
	A three-quarter sequence covering the basic theory of affine and projective 
	varieties, rings of functions, the Hilbert Nullstellensatz, localization, and 
	dimension; the theory of algebraic curves, divisors, cohomology, genus, and the 
	Riemann-Roch theorem; and related topics. 
\end{abstract}

\section{September 25, 2019}
The first thing that one asks is ``what is geometry?'' One needs to be able to answer this 
question before they define AG. One idea is that geometry is topology + structure.

\subsection{What is Geometry?}

\begin{ex}
	Exotic differentiable structures on a sphere. There are many different smooth structures,
	all of which are independent of the topology,

	$S^1\times S^1$ has infinitely many complex structures (remember the parallelograms)!
\end{ex}

How to you go about defining the geometry of a thing? One idea from manifolds: charts. These 
describe the local models and the interesting part is how this comes together to a whole space.

There is another idea to capture the ``local'' model of geometry that underlies modern algebraic geometry:
consider the map $\varphi:W\to W'\in\bbC\bbP^n$ and then say that this map is $C^\infty$ if and only if its coordinate
functions are. But the coordinate functions are problematic, so we can replace it with the following idea:

$\phi:W\to W'$ is $C\infty$ if and only if for all $C^\infty$ functions $f:W'\to \bbR$,
the composition
\[\varphi^\ast f=f\circ \varphi:W\to \bbR\]
is $C^\infty$.

To capture the manifold structure on $M$, it is equivalent to know the set of $C^\infty$ fucnctions $U\to \bbR$ for every open $U\subseteq M$.

\subsection{The Big Idea}
So then the idea we are talking away here is that \textit{geometry is in the functions} that exist on a particular space!

Fix a field $k$. 
\begin{defn}
	A \textbf{space with functions} is a topological space $X$ along with a collection (a $k$-algebra!) $\calO(U)$ of maps $U\to k$ for each open 
	$U\subseteq X$. 

	$\calO(U)$ are called \textbf{regular functions} and must satisfy:
	\begin{itemize}
		\item Given an open cover $U_\alpha$ of $U$, a function is regular if and only if its restrictions to each element of the cover is regular.
		\item If $f:U\to k$ is regular, then $D(f)=\{u\in U|f(u)\ne 0\}$ is an open set and $\frac{1}{f}\in\calO(D(f))$.
	\end{itemize}
\end{defn}

For the next time, try to think of as many examples of this as you can. Next time will be a mind blowing example of a variety.

\section{September 27, 2019}
\begin{prob}
	Do all the exercises in Kempf chapter 1!
\end{prob}

For now we assume that $k$ is algebraically closed.

\subsection{Examples of spaces with functions}
There were lots of suggestions, but here are a couple:
\begin{ex}
	Let $X=\bbS^2$ and let $\calO_X^{cts}$ be the continuous $\bbC$-valued functions. Alternatively
	we could consider a different sheaf $\calO_X^{an}$, the holomorphic functions. Or we could 
	consider $\calO_X^\infty$, the $C^\infty$ functions (under some smooth structure).
\end{ex}

\begin{defn}
	A \textbf{morphism} of spaces with functions between $(X,\calO_X)$ and $(Y,\calO_Y)$ is a continuous map 
	$\varphi:X\to Y$ such that for all $U\subseteq Y$ open and $f\in\calO_Y(U)$, the function 
	\[\phi^\ast f=f\circ\phi|_{\phi^{-1}(U)}:\phi^{-1}(U)\to k\in \calO_X(\phi^{-1}(U))\]

	In other words, a morphism of spaces with functions is a map of spaces that \textit{respects the regular functions.}
\end{defn}
\begin{ex}
	Let $X,Y$ be topological spaces and let $\calO_X$ and $\calO_Y$ be the continuous functions. Then morphisms are just continuous maps.
\end{ex}
\begin{ex}
	When $X$ and $Y$ are manifolds and $\calO_\bullet$ are complex-valued functions, then the maorphisms are maps 
	of manifolds.
\end{ex}

So now we return to the examples we saw before: $(\bbS^2,\calO^\infty)$, $(\bbS^2,\calO^{cts})$, and $(\bbS^2,\calO^{an})$.
A natural question to ask is when we have morphisms between these spaces to see if there exist ones that are the identity on $\bbS^2$.

Consider the identity map from the continuous to the analytic functions. Then take any map $f\in\calO^{an}$ and consider that 
\[f=f\circ id_{id^{-1}(U)}:U\to k\in\calO^{cts}(U)\]
and there is no map in the other direction.
\begin{rmk}
	Notice that since we are pulling functions back, the maps go in the opposite direction as you may think at first.
\end{rmk}

We can also talk about \textbf{open subspaces}. If $V\subseteq X$ is an open subset, we can let the induced space with functions 
be $(V,\calO_V)$ where if $U\subseteq V$ then $\calO_V(U):=\calO_X(U)$.

\subsection{Varieties}
\begin{defn}
	An \textbf{affine $k$-variety} is a space with functions $(Y,\calO_Y)$ such that for every space with functions $(X,\calO_X)$,
	the natural map 
	\[\Hom((X,\calO_X),(Y,\calO_Y))\to\Hom_{\Alg}(\calO_Y(Y),\calO_X(X))\]
	is a bijection and furthermore $\calO_Y(Y)=:k[Y]$ is a finitely generated $k$-algebra.
\end{defn}
\begin{rmk}
	The idea here is that the algebra maps (on the right) are precisely the same as the geometry maps (on the left).
	Algebraic geometry, baby.
\end{rmk}
So then this leads to a very simple (loose) definition:
\begin{defn}
	A \textbf{variety} is something that is covered by affine varieties.
\end{defn}

\begin{ex}
	$\bbA^1=k$. Give this space the cofinite topology. Then if we have $U=k\setminus\{x_1,\dots,x_n\}\subset \bbA^1$,
	\[\calO_{\bbA^1}(U)=\{f(t)\in k(t)| \text{poles are in }\{x_i\}\}\]
\end{ex}
\begin{prob}
	Show that $\bbA^1$ is an affine variety!
\end{prob}
\begin{rmk}
	Notice that this staetment is equivalent to saying that any morphism of spaces with functions gives us a regular map $X\to k$.
\end{rmk}



\end{document}