\documentclass[12pt]{article}

\usepackage{setspace}

\usepackage{amsmath, graphicx, color, fancyhdr, tikz-cd, mdframed, enumitem, framed, adjustbox, bbm, upgreek, xcolor, hyperref, manfnt}
\usepackage[framed,thmmarks]{ntheorem}
\usepackage[style=alphabetic, bibencoding=utf8]{biblatex}
%Set the bibliography file
\bibliography{sources}

\usepackage[T1]{fontenc}
\usepackage[urw-garamond]{mathdesign}
\usepackage{garamondx}

%Replacement for the old geometry package
\usepackage{fullpage}

%Input my definitions
/home/nico/latex-includes/mydefs.tex

%Shade definitions
\theoremindent0cm
\theoremheaderfont{\normalfont\bfseries} 
\def\theoremframecommand{\colorbox[rgb]{0.9,1,.8}}
\newshadedtheorem{defn}[thm]{Definition}

%%%%%%%%%%%%%%%%%%%%%%%%%%%%%%%%%%%%%%%%%%%%%%%%%%%%%%%%%%%%%%%%%%%%%%
%%%%%%%%%%%%%%%%%%%%%%% Customize Below %%%%%%%%%%%%%%%%%%%%%%%%%%%%%%
%%%%%%%%%%%%%%%%%%%%%%%%%%%%%%%%%%%%%%%%%%%%%%%%%%%%%%%%%%%%%%%%%%%%%%

%header stuff
\setlength{\headsep}{24pt}  % space between header and text
\pagestyle{fancy}     % set pagestyle for document
\lhead{Notes on the Schur-Weyl Functor} % put text in header (left side)
\rhead{Nico Courts} % put text in header (right side)
\cfoot{\itshape p. \thepage}
\setlength{\headheight}{15pt}
%\allowdisplaybreaks

% Document-Specific Macros
\DeclareMathOperator{\1}{\mathbbm{1}}
\DeclareMathOperator{\GL}{GL}
\DeclareMathOperator{\Sym}{Sym}

\begin{document}
%make the title page
\title{Algebraic Geometry\vspace{-1ex}}
\author{A course by Max Lieblich\\
Notes by Nico Courts}
\date{Autumn 2019/ Winter and Spring 2020}
\maketitle

\begin{abstract}
	A three-quarter sequence covering the basic theory of affine and projective 
	varieties, rings of functions, the Hilbert Nullstellensatz, localization, and 
	dimension; the theory of algebraic curves, divisors, cohomology, genus, and the 
	Riemann-Roch theorem; and related topics. 
\end{abstract}

\section{September 25, 2019}
The first thing that one asks is ``what is geometry?'' One needs to be able to answer this 
question before they define AG. One idea is that geometry is topology + structure.

\subsection{What is Geometry?}

\begin{ex}
	Exotic differentiable structures on a sphere. There are many different smooth structures,
	all of which are independent of the topology,

	$S^1\times S^1$ has infinitely many complex structures (remember the parallelograms)!
\end{ex}

How to you go about defining the geometry of a thing? One idea from manifolds: charts. These 
describe the local models and the interesting part is how this comes together to a whole space.

There is another idea to capture the ``local'' model of geometry that underlies modern algebraic geometry:
consider the map $\varphi:W\to W'\in\bbC\bbP^n$ and then say that this map is $C^\infty$ if and only if its coordinate
functions are. But the coordinate functions are problematic, so we can replace it with the following idea:

$\phi:W\to W'$ is $C\infty$ if and only if for all $C^\infty$ functions $f:W'\to \bbR$,
the composition
\[\varphi^\ast f=f\circ \varphi:W\to \bbR\]
is $C^\infty$.

To capture the manifold structure on $M$, it is equivalent to know the set of $C^\infty$ fucnctions $U\to \bbR$ for every open $U\subseteq M$.

\subsection{The Big Idea}
So then the idea we are talking away here is that \textit{geometry is in the functions} that exist on a particular space!

Fix a field $k$. 
\begin{defn}
	A \textbf{space with functions} is a topological spae $X$ along with a collection (a $k$-algebra!) $\calO(U)$ of maps $U\to k$ for each open 
	$U\subseteq X$. 

	$\calO(U)$ are called \textbf{regular functions} and must satisfy:
	\begin{itemize}
		\item Given an open cover $U_\alpha$ of $U$, a function is regular if and only if its restrictions to each element of the cover is regular.
		\item If $f:U\to k$ is regular, then $D(f)=\{u\in U|f(u)\ne 0\}$ is an open set and $\frac{1}{f}\in\calO(D(f))$.
	\end{itemize}
\end{defn}

For the next time, try to think of as many examples of this as you can. Next time will be a mind blowing example of a variety.


\end{document}