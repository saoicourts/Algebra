\documentclass[12pt]{article}

\usepackage{setspace}

\usepackage{amsmath, amsfonts, amssymb, graphicx, color, fancyhdr, lipsum, scalerel, stackengine, mathrsfs, tikz-cd, mdframed, enumitem, framed, adjustbox, bm, upgreek, xcolor, hyperref}
\usepackage[framed,thmmarks]{ntheorem}
\usepackage[style=alphabetic]{biblatex}
%Set the bibliography file
\bibliography{sources}

%Replacement for the old geometry package
\usepackage{fullpage}

%Input my definitions
/home/nico/latex-includes/mydefs.tex

%Shade definitions
\theoremindent0cm
\theoremheaderfont{\normalfont\bfseries} 
\def\theoremframecommand{\colorbox[rgb]{0.9,1,.8}}
\newshadedtheorem{defn}[thm]{Definition}

%%%%%%%%%%%%%%%%%%%%%%%%%%%%%%%%%%%%%%%%%%%%%%%%%%%%%%%%%%%%%%%%%%%%%%
%%%%%%%%%%%%%%%%%%%%%%% Customize Below %%%%%%%%%%%%%%%%%%%%%%%%%%%%%%
%%%%%%%%%%%%%%%%%%%%%%%%%%%%%%%%%%%%%%%%%%%%%%%%%%%%%%%%%%%%%%%%%%%%%%

%header stuff
\setlength{\headsep}{24pt}  % space between header and text
\pagestyle{fancy}     % set pagestyle for document
\lhead{Notes on Local Duality} % put text in header (left side)
\rhead{Nico Courts} % put text in header (right side)
\cfoot{\itshape p. \thepage}
\setlength{\headheight}{15pt}
\allowdisplaybreaks

% Document-Specific Macros
% Primes and Maximals for the lazy person.
\newcommand{\p}{\frakp}
\newcommand{\m}{\frakm}

\begin{document}
%make the title page
\title{Local Duality Theorems \vspace{-1ex}}
\author{Nico Courts}
\date{Summer 2019}
\maketitle

\renewcommand{\abstractname}{Introduction}
\begin{abstract}
	These notes are my attempt to understand the current state of the art in ``local duality'' theorems, especially 
	of the kind investigated by Benson, Iyengar, Krause, and Pevtsova (herein abbreviated BKIP) in \cite{BIKPgroupschemes} and \cite{BKIPgorenstein}.

	Furthermore I will continue my investigation into quantum groups and other Hopf algebras and see if some of the methods developed earlier 
	can be applied to this new area.
\end{abstract}

\section{Current State of the Art}
\subsection{Ideas to flush out}
From meeting with Julia, the idea that I have gathered is that these two papers constitute two (slightly) different approaches 
to showing that the notion of Serre duality (which by itself is a ``global'' phenomenon) restricts ($\frakp$)-locally (more on this later)
to an analogous result.

The original result in \cite{BIKPgroupschemes} proves that such a duality exists for finite group schemes, but the objects of study here are 
much simpler (for instance the algebras that arise are Frobenius). Furthermore (this part may be sketchy), the results that are proven 
rely on the (relatively simple) monoidal structure of $G$-modules (where $G$ is a finite group scheme). 

The newer result in \cite{BKIPgorenstein} reconstructs this result in the context of Gorenstein rings, which are decidedly less degenerate than 
the case of finite group schemes. What is important here is that this is done although the structure theory of these algebras 
is rather poorly understood, so what this represents is the formation of the idea that this duality somehow can be understood at
a higher level, meaning that it may apply to broader classes of algebras.

\subsection{Background results and definitions}
I have seen a good deal of these results already, but it will be useful to put everything into one place so I can reference them as needed.

First, Serre duality, which comes from \cite{hartshorneAG}:
\begin{thm}[Serre Duality]\label{thm-serre-duality}
	Let $XS$ be a projective Cohen-Macaulay scheme of equidimension $n$ over $k$. Then for any locally free sheaf $\scrF$ on $X$, 
	there are natural isomorphisms
	\[H^i(X,\scrF)\cong H^{n-i}(X,\scrF^\vee\otimes \omega_X^\circ)'\]
	where $(-)'$ indicates taking a vector space dual, $(-)^\vee$ indicates taking the dual of a locally free sheaf via the functor
	$\mathscr Hom(-,\calO_X)$ (sheaf hom) and $\omega_X^\circ$ denotes a dualizing sheaf (\ref{defn-dualizing-sheaf}) on $X$.
\end{thm}

\begin{defn}\label{defn-dualizing-sheaf}
	Let $X$ be a proper scheme of dimension $n$. Recall that a \textbf{dualizing sheaf} is a (coherent) sheaf $\omega_X^\circ$ such that there is an isomorphism 
	\[\Hom(\scrF,\omega_X^\circ)\cong H^n(X,\scrF)'.\]
\end{defn}
\begin{rmk}
	In fact, there is something more: that this isomorphism is induced from a natural pairing of $\Hom$ and $H^n$. I won't worry 
	about this too much for now, but it can be found in \cite[p. 240]{hartshorneAG}
\end{rmk}

I will also include the definition and explanation of compact generation that I wrote down for my Brown representability presentation.
\begin{defn}[Compact Object]\label{defn-compact-obj}
	Let $X\in\calT$ be an object in a triangulated category. Then $X$ is called \textbf{compact} if for every coproduct of objects in $\calT$
	\[\Hom_\calT\left(X,\coprod_{\lambda\in\Lambda}t_\lambda\right)=\coprod_{\lambda\in\Lambda}\Hom_\calT(X,t_\lambda)\]
\end{defn}
\begin{rmk}
	The idea here is that when the codomain is a product, this equality always exists. The projections give you 
	a product of maps into the factors and the universal property of products gives you a (unique!) map into the product if you
	have maps into each of the factors.

	This does not hold in general for coproducts, however! Recall that in additive categories 
	one has a canonical isomorphism between finite products and coproducts. Since triangulated 
	categories are additive, this is akin to saying that $X$ is well-behaved in that the maps 
	from $X$ still split even when the codomain is an arbitrarily large coproduct.
\end{rmk}
\begin{defn}[Compactly Generated Triangulated Category]\label{defn-compact-tricat}
	Let $\calT$ be a triangulated category. Then $\calT$ is called \textbf{compactly generated by a set $G$} of its elements 
	if $\calT$ is closed under small coproducts and $G$ consists of compact objects of $\calT$ such that for all $X\in\calT$,
	\[\Hom_\calT(G,X)=0\quad\Rightarrow\quad X=0.\]
\end{defn}
\begin{rmk}
	Here the generators of $\calT$ are elements that have ``sufficient complexity'' to detect all nonzero elements. 
	
	In the more general context where $\calT$ is not necessarily additive, we can generalize this definition to a set of objects 
	such that for every generator $E\in G$ and any $X,Y\in\calC$, if for every $f,g\in\Hom(X,Y)$ and $h\in\Hom(E,X)$ we have 
	\[f\circ e=g\circ e\]
	then 
	\[f=g\]
	and here we would say that $G$ has ``sufficient complexity'' to differentiate between any two non-identical maps.
\end{rmk}

\begin{defn}\label{defn-R-linear-cat}
	Recall that with such an $R$, an \textbf{$R$-linear category} $\calC$ is one such that the graded ring $\End^\ast_\calC(X)$ admits
	an $R$ module structure via a graded ring homomorphism $\varphi_X:R\to \End^\ast_\calC(X)$.
\end{defn}

The following comes from \cite{nlab-serre-functor}. 
\begin{defn}\label{defn-serre-func}
	Let $\calS$ be an endofunctor $\calS:\calA\to\calA$ on a $k$-linear category with finite dimensional $\Hom$ objects,
	where $k$ is any field. Then $\calS$ is called a \textbf{Serre functor} if it is an additive equivalence admitting 
	bi-functorial isomorphisms
	\[\phi_{A,B}:\Hom_\calA(A,B)\xrightarrow{\sim}\Hom_\calA(B,\calS(A))^\ast\]
	for all $A$ and $B$ in $\calA$.
\end{defn}



\subsection{BKIP Paper 1---Finite Group Schemes}
Let's start off with the primary result. We will need some definitions and other results to get us there, but it will give us a sense of 
where we're headed.
\subsubsection{Main Results}
\begin{thm}[BKIP `18]\label{thm-BKIP18}
	Let $G$ be a finite group scheme over $k$. Fix some homogeneous prime ideal $\frakp\lhd H^\ast(G,k)$ not containing $H^{\ge 1}(G,k)$. Let $C=\gamma_\frakp(\stmod G)$
	and let $d$ be the Krull dimension of $H^\ast(G,k)/\frakp$. Then the assignment
	\[M\mapsto \Omega^d\delta_G\otimes_k M\]
	induces a Serre functor (defn~\ref{defn-serre-func}) for $C$. Thus for all $M$ and $N$ in $C$,
	\[\Hom_{H^\ast(G,k)}(\Hom_C^\ast(M,N),I(\frakp))\cong\Hom_C(N,\Omega^d\delta_G\otimes M)\]
	where $I(\frakp)$ is the injective hull of the graded $H^\ast(G,k)$ module $H^\ast(G,k)/\frakp$.
\end{thm}

\noindent This result follows from a more general result:
\begin{thm}[BKIP `18]\label{thm-BKIP18-ext}
	For any $G$ module $M$ and $I\in\bbZ$, there is a natural isomorphism
	\[\widehat{\Ext}^i_G(M,\Gamma_\frakp(\delta_G))\cong\Hom_{H^\ast(G,k)}(H^{\ast-d-i}(G,M),I(\frakp)).\]
\end{thm}
\begin{rmk}
	Notice that here $\Gamma_\frakp(\delta_G)$ plays the part of the dualizing object in the second theorem, as Serre duality takes 
	on a similar form when $\scrF$ is a coherent sheaf on a nonsingular projective variety $X$ of dimension $n$:
	\[\Ext_X^i(\scrF,\omega_X)\cong \Hom_k(H^{n-i}(X,\scrF),k)\]
\end{rmk}

\subsubsection{Conventions}

Throughout we are fixing $R$, a \textbf{graded commutative, Noetherian} ring and $\calT$, a \textbf{compactly-generated (\ref{defn-compact-obj}, \ref{defn-compact-tricat}) 
$R$-linear (\ref{defn-R-linear-cat}) triangulated category.} Let $\calT^c$ be the full subcategory of $\calT$ comprising of compact objects.

\subsubsection{Torsion and Localization}
We should brush up on a bit of how torsion and localization from rings passes to such a $\calT$: We say that $M\in \Rmod$ is \textbf{$\fraka$-torsion} if 
$M_\frakq=0$ whenever $\fraka\not\subseteq\frakq$. In other words, whenever we invert something in $\fraka$, we kill everything. Analogously,
we say that $X\in\calT$ is $\fraka$-torsion if $\Hom^\ast_\calT(C,X)$ is $\fraka$-torsion for all $C\in\calT^c$.

Now if $\p$ is a (homogeneous!) prime in $R$,  we say that $M\in \Rmod$ is $\p$-local if the localization map $M\mapsto M_\frakp$ is invertible.
This means that ``all of the action is at $\p$'' -- recall that in the case of rings, this would mean that $\frakp$ is maximal and contains all non-units,
so $R$ is local. We say that $X\in\calT$ is $\p$-local if $\Hom^\ast_\calT(C,X)$ is $\p$-local for all $C\in\calT^c$.

We will be devoting significant attention to $\Gamma_\p(\calT)$, the (full, localizing) subcategory of $\calT$ consisting 
of the $\p$-local $\p$-torsion elements. As far as I can tell, we should be thinking of $R/\p^i$ as an analogy in $\Rmod$.

One of the primary functors in question (the one that extracts the ``local'' nature) is $\Gamma_\p$, which assigns to $\calT$ the 
set (actually it's a localizing subcategory in $\calT$)
\[\Gamma_\p(\calT)=\{X\in\calT| X\text{is $\p$-local and $\p$-torsion\}}.\]

The paper points this out as an important result:
\begin{lem}
	$\Gamma_\p(\StMod G)$ is finitely generated as a triangulated category by the family of objects $(M//\p)_\p$.
\end{lem}
Here $M//\p$ is defined as $M//p_1\otimes\cdots\otimes M//p_k$ where $p_1,\dots,p_k$ generate $\p$ and $M//a$ is defined to 
be the mapping cone of the map $f_a:k\to \Omega^{-d}k$ that comes from the element $a\in H^d(G,k)$.


%%%%%%%%%%%%%%%%%%%%%%%%%%%%%%%%%%%%%%
%%%%%%%%%%  Bibliography %%%%%%%%%%%%%
%%%%%%%%%%%%%%%%%%%%%%%%%%%%%%%%%%%%%%
\medskip

\printbibliography

\end{document}