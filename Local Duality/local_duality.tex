\documentclass[12pt]{article}

\usepackage{setspace}

\usepackage{amsmath, amsfonts, amssymb, graphicx, color, fancyhdr, lipsum, scalerel, stackengine, mathrsfs, tikz-cd, mdframed, enumitem, framed, adjustbox, bm, upgreek, xcolor, hyperref}
\usepackage[framed,thmmarks]{ntheorem}
\usepackage[style=alphabetic]{biblatex}
%Set the bibliography file
\bibliography{sources}

%Replacement for the old geometry package
\usepackage{fullpage}

%Input my definitions
/home/nico/latex-includes/mydefs.tex

%Shade definitions
\theoremindent0cm
\theoremheaderfont{\normalfont\bfseries} 
\def\theoremframecommand{\colorbox[rgb]{0.9,1,.8}}
\newshadedtheorem{defn}[thm]{Definition}

%%%%%%%%%%%%%%%%%%%%%%%%%%%%%%%%%%%%%%%%%%%%%%%%%%%%%%%%%%%%%%%%%%%%%%
%%%%%%%%%%%%%%%%%%%%%%% Customize Below %%%%%%%%%%%%%%%%%%%%%%%%%%%%%%
%%%%%%%%%%%%%%%%%%%%%%%%%%%%%%%%%%%%%%%%%%%%%%%%%%%%%%%%%%%%%%%%%%%%%%

%header stuff
\setlength{\headsep}{24pt}  % space between header and text
\pagestyle{fancy}     % set pagestyle for document
\lhead{Notes on Local Duality} % put text in header (left side)
\rhead{Nico Courts} % put text in header (right side)
\cfoot{\itshape p. \thepage}
\setlength{\headheight}{15pt}
\allowdisplaybreaks

% Document-Specific Macros


\begin{document}
%make the title page
\title{Local Duality Theorems \vspace{-1ex}}
\author{Nico Courts}
\date{Summer 2019}
\maketitle

\renewcommand{\abstractname}{Introduction}
\begin{abstract}
	These notes are my attempt to understand the current state of the art in ``local duality'' theorems, especially 
	of the kind investigated by Benson, Iyengar, Krause, and Pevtsova (herein abbreviated BKIP) in \cite{BIKPgroupschemes} and \cite{BKIPgorenstein}.

	Furthermore I will continue my investigation into quantum groups and other Hopf algebras and see if some of the methods developed earlier 
	can be applied to this new area.
\end{abstract}

\section{Current State of the Art}
\subsection{Ideas to flush out}
From meeting with Julia, the idea that I have gathered is that these two papers constitute two (slightly) different approaches 
to showing that the notion of Serre duality (which by itself is a ``global'' phenomenon) restricts ($\frakp$)-locally (more on this later)
to an analogous result.

The original result in \cite{BIKPgroupschemes} proves that such a duality exists for finite group schemes, but the objects of study here are 
much simpler (for instance the algebras that arise are Frobenius). Furthermore (this part may be sketchy), the results that are proven 
rely on the (relatively simply) monoidal structure of $G$-modules (where $G$ is a finite group scheme). 

The newer result in \cite{BKIPgorenstein} reconstructs this result in the context of Gorenstein rings, which are decidedly less degenerate than 
the case of finite group schemes. What is important here is that this is done although the structure theory of these algebras 
is rather poorly understood, so what this represents is the formation of the idea that this duality somehow can be understood at
a higher level, meaning that it may apply to broader classes of algebras.

\subsection{Background results and definitions}
I have seen a good deal of these results already, but it will be useful to put everything into one place so I can reference them as needed.

First, Serre duality, which comes from \cite{hartshorneAG}:
\begin{thm}
	Let $XS$ be a projective Cohen-Macaulay scheme of equidimension $n$ over $k$. Then for any locally free sheaf $\scrF$ on $X$, 
	there are natural isomorphisms
	\[H^i(X,\scrF)\cong H^{n-i}(X,\scrF^\vee\otimes \omega_X^\circ)'\]
	where $(-)'$ indicates taking a vector space dual, $(-)^\vee$ indicates taking the dual of a locally free sheaf:
	$\mathscr Hom(-,\calO_X)$ (sheaf hom) and $\omega_X^\circ$ denotes a dualizing sheaf on $X$.
\end{thm}

\begin{defn}
	Let $X$ be a proper scheme of dimension $n$. Recall that a dualizing sheaf is a (coherent) sheaf $\omega_X^\circ$ such that there is an isomorphism 
	\[\Hom(\scrF,\omega_X^\circ)\cong H^n(X,\scrF)'.\]
\end{defn}
\begin{rmk}
	In fact, there is something more: that this isomorphism is induced from a natural pairing of $\Hom$ and $H^n$. I won't worry 
	about this too much for now, but it can be found in \cite[p. 240]{hartshorneAG}
\end{rmk}

\subsection{BKIP Paper 1---Finite Group Schemes}
Let's start off with the primary result. We will need some definitions and other results to get us there, but it will give us a sense of 
where we're headed.
\begin{thm}[BKIP) `18]

\end{thm}

%%%%%%%%%%%%%%%%%%%%%%%%%%%%%%%%%%%%%%
%%%%%%%%%%  Bibliography %%%%%%%%%%%%%
%%%%%%%%%%%%%%%%%%%%%%%%%%%%%%%%%%%%%%
\medskip

\printbibliography

\end{document}