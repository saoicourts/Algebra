\documentclass[12pt]{article}

\usepackage{setspace}

\usepackage{amsmath, amsfonts, amssymb, graphicx, color, fancyhdr, lipsum, scalerel, stackengine, mathrsfs, tikz-cd, mdframed, enumitem, framed, adjustbox, bm, upgreek, x	color}
\usepackage[framed,thmmarks]{ntheorem}
\usepackage[mathscr]{euscript}

%set up theorem/definition/etc envs
%Problems will be created using their own counter and style
\theoreminframepreskip{0pt}
\theoreminframepostskip{0pt}
\newframedtheorem{prob}{Problem}[part]
\renewcommand\theprob{\arabic{part}.\arabic{prob}}

%solution template
\theoremstyle{nonumberbreak}
\theoremindent0.5cm
\theorembodyfont{\upshape}
\theoremseparator{:}
\theoremsymbol{\ensuremath\spadesuit}
\newtheorem{sol}{Solution}

%Theorems, Lemmas, and Corollaries
\theoremstyle{changebreak}
\theoremseparator{}
\theoremsymbol{}
\theoremindent0.5cm
\theoremheaderfont{\color{violet}\bfseries} 

\newtheorem{thm}{Theorem}[subsection]
\theoremheaderfont{\bfseries}
\newtheorem{lem}[thm]{Lemma}
\newtheorem{cor}[thm]{Corollary}

%Create a new env that references a theorem and creates a 'primed' version
%Note this can be used recursively to get double, triple, etc primes
\newenvironment{thm-prime}[1]
  {\renewcommand{\thethm}{\ref{#1}$'$}%
   \addtocounter{thm}{-1}%
   \begin{thm}}
  {\end{thm}}

\setlength\fboxsep{15pt}

%Shade definitions
\theoremindent0cm
\theoremheaderfont{\normalfont\bfseries} 
\def\theoremframecommand{\colorbox[rgb]{.9,.8,1}}
\newshadedtheorem{defn}[thm]{Definition}

%Man, that's really good! Let's use the same thing for definitons.
\newenvironment{def-prime}[1]
  {\renewcommand{\thethm}{\ref{#1}$'$}%
   \addtocounter{thm}{-1}%
   \begin{def}}
  {\end{def}}

%proofs
\theoremstyle{nonumberbreak}
\theoremindent0.5cm
\theoremheaderfont{\sc}
\theoremseparator{}
\theoremsymbol{\ensuremath\spadesuit}
\newtheorem{prf}{Proof}

%remarks
\theoremstyle{change}
\theoremindent0.5cm
\theoremheaderfont{\sc}
\theoremseparator{:}
\theoremsymbol{}
\newtheorem{rmk}[thm]{Remark}

%Replacement for the old geometry package
\usepackage{fullpage}

%Put page breaks before each part
\let\oldpart\part%
\renewcommand{\part}{\clearpage\oldpart}%

%Center each figure by default
\makeatletter
\g@addto@macro\@floatboxreset{\centering}
\makeatother

%header stuff
\setlength{\headsep}{24pt}  % space between header and text
\pagestyle{fancy}     % set pagestyle for document
\lhead{Notes on Lie Algebras} % put text in header (left side)
\rhead{Nico Courts} % put text in header (right side)
\cfoot{\itshape p. \thepage}
\setlength{\headheight}{15pt}
\allowdisplaybreaks

%Set of Integers
\newcommand*{\Z}{
\mathbb{Z}
}
%Set of Natural Numbers
\newcommand*{\N}{
\mathbb{N}
}
%Set of Real Numbers
\newcommand*{\R}{
\mathbb{R}
}
%Set of Complex Numbers
\newcommand*{\C}{
\mathbb{C}
}
%Rationals
\newcommand*{\Q}{
\mathbb{Q}
}

%Section break
\newcommand*{\brk}{
\rule{2in}{.1pt}
}

\DeclareMathOperator{\Aut}{Aut}

%raise that Chi!
\DeclareRobustCommand{\Chi}{{\mathpalette\irchi\relax}}
\newcommand{\irchi}[2]{\raisebox{\depth}{$#1\chi$}} 

%Image
\DeclareMathOperator{\im}{Im}

%Coker
\DeclareMathOperator{\coker}{coker}

%characteristic
\DeclareMathOperator{\ch}{char}

%rank
\DeclareMathOperator{\rank}{rank}

%identity map
\DeclareMathOperator{\id}{id}

%Lie algebra stuff
\DeclareMathOperator{\gl}{\mathfrak{gl}}
\let\sl\relax
\DeclareMathOperator{\sl}{\mathfrak{sl}}
\DeclareMathOperator{\tr}{tr}

%fix tilde
\let\tilde\relax
\newcommand*{\tilde}[1]{\widetilde{#1}}

% Enumerate will automatically use letters (e.g. part a,b,c,...)
\setenumerate[0]{label=(\alph*)}

\begin{document}
%make the title page
\title{Lie Algebras and Groups\vspace{-1ex}}
\author{A course by: Monty McGovern\\
Notes by: Nico Courts}
\date{Winter 2019}
\maketitle

\renewcommand{\abstractname}{Introduction}
\begin{abstract}
	These notes are my best attempt at following along with our \textit{Math 508 --
	Lie Algebras} course at UW. This is my first time trying to type my 
	notes on-the-fly in class so we'll see how well this goes. The course reference
	is Humphreys' \textit{Introduction to Lie Algebras and Representation Theory.}

	The course description follows:
	
	\brk

	This is the second course in the Algebraic Structures sequence. I will classify 
	finite-dimensional complex semisimple Lie algebras, also proving some structural 
	results on general Lie algebras along the way. Although one usually first 
	encounters Lie algebras in a manifolds course, the treatment (following the text) 
	will be entirely algebraic.
\end{abstract}

\section{January 7, 2019}
The homework is posted on Monty's website. :) 
\subsection{Lie algebras}

This course will be studying Lie algebras, but as opposed to their treatment in manifolds, 
we will be studying them from a purely algebraic point of view. The book (Humphreys)
actually never defines a Lie group.

\begin{defn}
	A \textbf{Lie Algebra} $L$ or $\mathfrak{g}$ over a field $k$ is a $k$-vector space (usually f.d.)
	along with a \textit{bracket operation} $[vw]:L\times L\to L$ such that $[\cdot\cdot]$ is 
	\begin{itemize}
		\item anticommutative,
		\item bilinear,
		\item $[x[yz]]=[[xy]z]+[y[xz]]$
	\end{itemize}
\end{defn}

\begin{rmk}
	The last principle above is actually equivalent to the \textit{Jacobi identity:}
	\[[x[yz]]+[y[xz]]+[z[xy]].\]
	This follows from bilinearity and anticommutativity of the bracket.
\end{rmk}
The most natural place for these to arise is as \textit{derivations} on an algebra!
\begin{defn}
	A \textbf{$k$-derivation} $d:A\to A$ on an algebra $A$ over $k$ is a $k$-linear map
	satisfying the Leibniz rule.
\end{defn}
\begin{rmk}
	Some key facts about derivations (for us):
	\begin{itemize}
		\item Given a fixed $a\in A$, the map $d_a$ sending $b\mapsto ab-ba$, the \textbf{commutator}
		$[ab]$ is a derivation.
		\item If $d,e$ are derivations, then so is $[de]=de-ed$, where $de$ is the \textit{composite}
		of $d$ and $e$ as opposed to the product.
	\end{itemize}
\end{rmk}

\subsection{Examples}

A main source of Lie algebras is (associative) algebras! \textit{Any associative $k$-algebra $A$}
becomes a Lie algebra over $k$, taking $[ab]=ab-ba.$ In particular, one obvious choice for $k$-algebra
is $M_n(k)=\gl_n(k)$, the (Lie) algebra of $n\times n$ matrices over $k$.

\textbf{Lie subalgebras} are what you'd expect (including closure under brackets). Notice
that if $L'\le L$, then they \textbf{must both be over the same field.}

If $L$ is a $k$-Lie algebra and $I\lhd L$ is an ideal of $L$, then the quotient space $L/I$ becomes a 
Lie algebra with $[x+I,y+I]=[xy]+I$ as the bracket.

A \textbf{Lie algebra homomorphism} is a map $\varphi:L\to L'$ such that $\varphi$ is $k$-linear
and $\varphi([xy])=[\varphi(x)\varphi(y)].$

We get the usual first isomorphism theorem $L/\ker\varphi\cong \varphi(L).$

\brk

Associative algebras are not the only source of Lie algebras, however! One example is 
$\sl(n,k)=\{n\times n\text{ matrices over } k\text{ with trace zero}\}$

Note that this is \textbf{not closed under product} since $\tr(AB)\ne\tr A\tr B$ but $\tr(AB)=\tr(BA)$
so $\tr(AB-BA)=\tr(AB)-\tr(BA)=0$.
\begin{defn}
	We call this algebra (or, in fact any subalgebra of $\gl(n,k)$) \textbf{linear}. Think 
	``Linear'' means ``of matrices.''
\end{defn}

We say that $\sl(n,k)$ has \textbf{type} $A_{n-1}$. Eventually we will see seven types
$A-G$ of semisimple Lie algebras. The shift in index will emerge later.

$\sl(n,k)$ is, in fact, a simple Lie algebra: for $k=\C$, $\sl(n,\C)$ has no ideals
apart from the trivial ones. 

\brk

Other non-associative examples include $k^n$ with a bilinear form $(\cdot,\cdot)$ which
is either symmetric or skew-symmetric and (in either case) is nondegenerate.
\begin{defn}
	$(\cdot,\cdot)$ is \textbf{nondegenerate} if the map $v\mapsto (v,\cdot)$ is injective. Equivalently
	there is no $v\in V$ such that $(v,w)=0$ for all $w\in V$.
\end{defn}

Given $V=k^n$ and a bilinear form on $V$, we can look at all $X\in \gl(n,k)=\gl(V)$ such that 
$(Xv,w)=(v,Xw)$. Then $X$ is \textbf{skew-adjoint} with respect to the form. One can check that 
$[XY]$ is skew-adjoint whenever both $X$ and $Y$ are.

\subsection{Generating (skew) symmetric forms}
It ends up that the dot product (which a symmetric form) is misleadingly simple -- thus
we will look elsewhere.

If $M\in \gl(n,k)$ is symmetric, so that $M^t=M$, then $(v,w)=v^tMw$ is a symmetric. If 
instead $M$ is skew-symmetric, then the same definition yields a skew-symmetric form. 
This actually induces a one-to-one correspondence between matrices and forms.

In both cases, if $M$ is invertible, then the form will be nondegenerate. As a consequence, 
since skew-symmetric matrices are always singular in odd dimensions, we see that 
nondegenerate skew-symmetric forms (over $\ch k\ne 2$ where the two families of forms
coincide) exist only in even dimensions.

\subsection{A peek at classifications}
If we have a nondegenerate symmetric form where $n=2m$ is going to give us an algebra
of type $D_m$. If $n=2m+1$, then it is of type $B_m$. Both of these cases are called
\textbf{orthogonal.}

If instead we have a skew-symmetric form and $n=2m$, then this is of type $C_m$, and we 
call this algebra \textbf{symplectic.}

\brk

We will make a particular choice for our matrix $M$ and then study the resulting
Lie algebras in much more detail next time. The choices will be: 
\begin{itemize}
	\item For type $D_m$:
	\[\begin{pmatrix}
		0 & I_m\\
		I_m & 0
	\end{pmatrix}\]
	\item For type $C_m$:
	\[\begin{pmatrix}
		0 & -I_m\\
		I_m & 0
	\end{pmatrix}\]
	\item For type $B_m$:
	\[\begin{pmatrix}
		1 & 0 & 0\\
		0 & 0 & I_m\\
		0 & I_m & 0
	\end{pmatrix}\]
\end{itemize}
\end{document}
