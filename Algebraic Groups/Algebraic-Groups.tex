\documentclass[12pt]{article}

\usepackage{setspace}

\usepackage{amsmath, graphicx, color, fancyhdr, tikz-cd, mdframed, enumitem, framed, adjustbox, bbm, upgreek, xcolor, hyperref, manfnt}
\usepackage[framed,thmmarks]{ntheorem}
\usepackage[style=alphabetic, bibencoding=utf8]{biblatex}
%Set the bibliography file
\bibliography{sources}

\usepackage[T1]{fontenc}
\usepackage[urw-garamond]{mathdesign}
\usepackage{garamondx}

%Replacement for the old geometry package
\usepackage{fullpage}

%Input my definitions
/home/nico/latex-includes/mydefs.tex

%Shade definitions
\theoremindent0cm
\theoremheaderfont{\normalfont\bfseries} 
\def\theoremframecommand{\colorbox[rgb]{0.9,1,.8}}
\newshadedtheorem{defn}[thm]{Definition}

%%%%%%%%%%%%%%%%%%%%%%%%%%%%%%%%%%%%%%%%%%%%%%%%%%%%%%%%%%%%%%%%%%%%%%
%%%%%%%%%%%%%%%%%%%%%%% Customize Below %%%%%%%%%%%%%%%%%%%%%%%%%%%%%%
%%%%%%%%%%%%%%%%%%%%%%%%%%%%%%%%%%%%%%%%%%%%%%%%%%%%%%%%%%%%%%%%%%%%%%

%header stuff
\setlength{\headsep}{24pt}  % space between header and text
\pagestyle{fancy}     % set pagestyle for document
\lhead{Algebraic Groups} % put text in header (left side)
\rhead{Notes by Nico Courts} % put text in header (right side)
\cfoot{\itshape p. \thepage}
\setlength{\headheight}{15pt}
%\allowdisplaybreaks

% Document-Specific Macros
\DeclareMathOperator{\1}{\mathbbm{1}}
\DeclareMathOperator{\GL}{GL}
\DeclareMathOperator{\Sym}{Sym}
\DeclareMathOperator{\Spec}{Spec}

\begin{document}
%make the title page
\title{Algebraic Groups\vspace{-1ex}}
\author{A course by Jarod Alper and Julia Pevtsova\\
Notes by Nico Courts}
\date{Autumn 2019/ Winter and Spring 2020}
\maketitle

\begin{abstract}
	The topic of algebraic groups is a rich subject combining both group-theoretic and algebro-geometric-theoretic techniques. Examples include the general linear group $GL_n$, 
	the special orthogonal group $SO_n$ or the symplectic group $Sp_n$. Algebraic groups play an important role in algebraic geometry, representation theory and number theory.

	In this course, we will take the functorial approach to the study of linear algebraic groups (more generally, affine group schemes) equivalent to the study of Hopf algebras. 
	The classical view of an algebraic group as a variety will come up as a special case of a smooth algebraic group scheme. Our algebraic approach will be independent (even complementary) to the analytic approach taken in the course on Lie groups.
\end{abstract}

\section{September 25, 2019}
\subsection{Group objects}
Let $\calC$ be a category with a final object and finite products.
\begin{defn}
	A \textbf{group object $G$ in $\calC$} is an object in $\calC$ along with multiplication, identity, and inverse morphisms
	satisfying the usual axioms.
\end{defn}

One thing is that we are using that there is a final object $\ast$ along with our identity morphism $e:\ast\to G$.
Here Jarrod explictly used the fact that there is a unique map to $\ast$.

\begin{ex}
	If $\calC$ is $\operatorname{Set}$, then $G$ is a group. If $\calC=\operatorname{Top}$, then $G$ is a topological group, smooth manifolds give Lie groups, and finally (interesting to us):
\end{ex}
\begin{defn}
	Let $S$ be a scheme and let $\calC$ be the category of schemes over $S$. Then a group object $G$ in $\calC$ is 
	a \textbf{group scheme over $S$.}
\end{defn}

WHen $k$ is a field and $\calC$ is schemes of finite type over $k$, we get a group scheme of finite type over $k$. There is not a great consensus on what makes an \textbf{algebraic group}, 
but this is what we will use.

When we instead restrict to \textit{affine schemes} we get an affine groupe scheme of finite tipe over $k$, or a \textbf{linear algebraic group.}

\subsection{Examples}
$\bbG_m=\operatorname{Spec} k[t]_t$ is one. 

If we consider the map $f:\bbG_m\to \bbG_m$ which on the level of elements sends $t\mapsto t^p$, the kernel is 
\[\mu_p=\ker(f)=\operatorname{Spec}k[t]/(t^p-1)\]
and that's great, but when $\ch k=p$, this causes the group scheme to be \textbf{unreduced}. This is (apparently) a case when you need to use schemes.

\subsection{The Functorial Approach}
Let $\calC$ be a category with object $X$. Define the functor $h_X:\calC^{op}\to \mathbf{Set}$ where 
\[h_X(Y)=\Hom_\calC(Y,X).\]

Then we have 
\begin{lem}[Yoneda]
	Let $G:\calC^{op}\to\mathbf{Set}$ be a functor. There is a natural bijection
	\[G(X)\simeq \operatorname{Nat}(h_X,G).\]
\end{lem}
\begin{prop}
	A group object $G$ in $\calC$  is the same as an aobject $X\in\calC$ together with a choice of factorization of 
	$h_X:\calC\to\mathbf{Set}$ through $\mathbf{Grp}$.
\end{prop}

\subsection{Exercises}
\begin{enumerate}
	\item Spell out all the details of the proof of the above propositon.
	\item Given a group object $G$, define in two ways what it means for it to act on another object. (In coordinates and functorially).
\end{enumerate}

\subsection{Some Interesting Facts}
If we had to write down five results that we'd like to get out of this class:
\begin{prop}
	Every affine group scheme of finite type over a field embeds into $GL_n$ as a closed subgroup.
\end{prop}
\begin{thm}[Chevalley's Theorem]
	Let $G$ be a finite type group scheme over a field. Then it factors as 
	\[1\to H\to G\to A\to 1\]
	where $A$ is abelian and $H$ is affine (linear algebraic).
\end{thm}
\begin{prop}
	If $G$ is an affine group scheme of finite type over $k$, then we have af actorization
	\[1\to U\to G\to R\to 1\]
	where $U$ is unipotent and $R$ is reductive.
\end{prop}
\begin{prop}
	$H\subseteq G$  a subgroup scheme. Then $G/H$ is a projective scheme.
\end{prop}
Finally we want to talk about Tanakka duality and how the representations of $G$ define $G$ itself.

\section{September 27th, 2019}
Last time we defined a group scheme (a group object in the category of schemes over a base scheme). We also mentioned that 
You could define it as a map $h_G:\mathbf{Sch}/S\to \mathbf{Set}$ along with a factorization through $\mathbf{Grp}$.

We defined an \textbf{algebraic group} over $k$ as a group scheme over $\operatorname{Spec} k$ of finite type and a \textbf{linear algebraic group}
to be an \textit{affine} group scheme over $k$ of finite type.

\subsection{Hopf Algebras}
Let $G=\Spec A$ be a linear algebraic group over $k$. I have seen most of these before (see Waterhouse or my Hopf algebra notes)
\begin{rmk}
	One think I haven't seen explicitly before: Notice that the augmentation ideal $\ker \varepsilon$, where $\varepsilon$ is the counit,
	is the (maximal!) ideal corresponding in the algebro-geometric sense to the identity element in $G$.
\end{rmk}
\begin{defn}
	A \textbf{Hopf algebra} is ...
\end{defn}
\begin{defn}
	Let $G$ be an algebraic group over $k$. Then if $h_G$ factors through $\Ab$, $G$ is called \textbf{commutative.}
\end{defn}

\subsection{Some Examples}
\begin{rmk}
	Note that to define a functor from schemes over $k$, is suffices to define it on affine schemes, thereby defining 
	the (Zariski) local behavior of any such map. Thus we really only need to consider maps in $\Alg$.
\end{rmk}
\begin{itemize}
	\item $\bbG_a$. Here we can define it as a functor that sends $S\mapsto\Gamma(S,\calO_S)$. Geometrically, $\bbG_a=\bbA^1$ where the multiplication is addition, inverses send $x\mapsto -x$ and the unit is the zero map.
	The Hopf algebraic picture is the usual dual thing.
	\item $\bbG_m$ as a scheme isthe map $S\mapsto \Gamma(S,\calO_S)^\ast$. In the geometric picture, $\bbA^1\setminus\{0\}$ and the algebra structure comes from multiplciation. Hopf is pretty easy.
	\item $\GL_n$ is a scheme that sends
	\[S\mapsto \left\{A=(a_{ij}): a_{ij}\in\Gamma(S,\calO_S), \det(A)\in\Gamma(S,\calO_S)^\ast\right\}\]
	the algebra is $\bbA^{n\times n}\setminus \{\det = 0\}$ with the usual multiplication. The coalgebra structure can be seen in the book.
\end{itemize}
This one requires some more explaination so I am setting it apart.
\begin{ex}
	Let $V$ be a finite dimensionatl vector space over $k$. Then we can define the algebraic group $V_a$ which sends 
	\[S\mapsto \Gamma(S,\calO_S)\otimes_k V.\]
	Geometrically we are looking at $\bbA(V)=\Spec\Sym ^\ast V^\vee\simeq \Spec k[x_1,\dots,x_n]$ where $n=\dim V$.
\end{ex}

What about finite groups? As a scheme, we want $G=\sqcup_{g\in G}\Spec k$. The functor sends $S\mapsto \operatorname{Mor}_{\operatorname{Set}}(\pi_0(S),G)$,
or maps from the connected components into $G$.

\begin{ex}
	Now consider the $n^{th}$ roots of unity: as a scheme, $\mu_n=\Spec k[t]/(t^n-1)\subseteq \bbG_m$.
	If both $k=\bar k$ and $\ch k\nmid n$, then $\mu_n\cong\bbZ/n\bbZ$.

	But if (e.g.) $k=\bbQ$, then $\mu_3$ is $\bbQ[t]/(t^3-1)=\Spec\bbQ\sqcup \Spec\bbQ(\xi)$ where $\xi$ is a primitive third root of unity.

	If, on the other hand, $k=\bar \bbF_3$ and consider $\mu_3$, we get a single point with residye field $\bar\bbF_3$.
\end{ex}

\begin{ex}
	If we are in the case of positive characteristic, then we get an algebraic group $\alpha_p$. Here the scheme is $\Spec k[x]/x^p$ and functorially it 
	is the map $S\mapsto \{F\in\Gamma(S,\calO_S)|f^p=0\}$.
\end{ex}

\subsection{Matrix Groups}
We already defined $\GL_n$, but we can also define 
\[\operatorname{SL}_n:S\mapsto\{A=(a_{ij})|\det A=1\}\]
with scheme $\Spec k[x_{ij}]/(\det-1)$.

We also have the (upper) triangular matrices $T_n$ and unitary group $U_n$ and diagonal group $D_n$

\begin{defn}
	Let $G$ be a linear algebraic group. Then 
	\begin{itemize}
		\item $G$ is a \textbf{vector group} if $G\cong V_a$ for some finite dimensional $V$.
		\item $G$ is a \textbf{split torus} if $G\cong \bbG_m^n$.
		\item $G$ is a \textbf{torus} if there is a field extention $k\to k'$ such that 
		\[G\times_{\Spec k}\Spec k'\cong \bbG^n_{m,k'}\]
	\end{itemize}
\end{defn}
\end{document}